\documentclass[aspectratio=169]{beamer}
\usepackage{borelian}

\begin{document}
  \classtitle{4}{Teoremas fundamentales y paradigmas estadísticos}{6 de enero de 2026}

  \begin{frame}{Estadística frecuentista}
    La estadística frecuentista es un enfoque que busca describir las probabilidades asociadas a un evento como la frecuencia relativa de ocurrencia de dicho evento en un conjunto grande de experimentos o ensayos repetidos bajo las mismas condiciones.

    \pause
    \begin{exampleblock}{Ejemplo}
      ¿Cuál es la probabilidad de lanzar un dado y obtener un número par en el enfoque frecuentista?

      \pause
      Para responder esta pregunta, imaginemos que podemos lanzar infinitamente el dado. Lo que se esperaría es que aproximadamente la mitad de los lanzamientos resulten en un número par ($2$, $4$, o $6$), si es que es un dado normal. Así, $\mathbb{P}(X \in \{2, 4, 6\}) = \frac{3}{6} = 0.5$.
    \end{exampleblock}
  \end{frame}

  \begin{frame}{Ley de los grandes números}
    En el ejemplo anterior, ``esperamos'' que la frecuencia relativa de obtener un número par se acerque a $\frac{1}{2}$ a medida que el número de lanzamientos del dado aumenta, pero ¿es esto realmente así?

    \pause
    \begin{block}{Rigurosidad matemática}
      En las matemáticas, cuando se tiene una creencia que no está probada ni refutada, se dice que es una \emph{conjetura}.
    \end{block}

    \pause
    La apuesta que realizamos es una conjetura. Pueden probarlo ustedes mismos. Lancen un dado muchas veces y calculen la frecuencia relativa de obtener un número par. ¿Se acerca a $\frac{1}{2}$? Ahora... ¿qué herramienta matemática me asegura que ese es el resultado correcto para la probabilidad?
  \end{frame}

  \begin{frame}{Ley de los grandes números}
    Por este motivo, se enuncia la \emph{ley de los grandes números}. Esta ley establece que, a medida que el número de ensayos independientes de un experimento aumenta, la frecuencia relativa de un evento específico converge hacia la probabilidad teórica de ese evento.
  \end{frame}

  \begin{frame}{Aplicación: ``la casa siempre gana''}
    Supongamos que mañana iremos a un casino como curso a jugar la ruleta, y decidimos apostar todo al rojo con los ingresos de la EdV (nadie fue obligado).

    \pause
    Consideremos que en una ruleta común y corriente, existen \textcolor{red}{$18$ números rojos}, $18$ números negros, y \textcolor{green!50!black}{$2$ números verdes} (el \textcolor{green!50!black}{$0$} y el \textcolor{green!50!black}{$00$}).

    \pause
    \begin{block}{Preguntas abiertas}
      \begin{itemize}
        \item ¿Por qué podemos estar \underline{ultraseguros} de que la casa siempre gana a largo plazo?

        \pause
        \item ¿Entonces nos han estafado mostrándonos a esas personas que se hacen millonarias?
      \end{itemize}
    \end{block}

    \pause
    De nuevo, es importante el pensamiento crítico.
  \end{frame}
\end{document}