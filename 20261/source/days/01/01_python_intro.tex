\documentclass[11pt,t,aspectratio=169]{beamer}
\usepackage{borelian}
\usepackage{minted}

\begin{document}
\classtitle{1}{Introducción a Python}{5 de enero de 2026}

%%%%%%
\begin{frame}{Contenidos de hoy}
    


    \begin{itemize}%[<+->]
        \item ¿Por qué Python en IA?

        %\item Ejemplos de Uso.

        \item Basicos de Python.
        
        \item Tipos de Datos en Python (Scalar Types).
        
        \item Condicionales \& Loops (Control Flow).
        
        \item Estructuras de Datos, Funciones y Archivos.
        
        \item Paquetes Esenciales para el Analisis de Datos.
    \end{itemize}
    
\end{frame}

\begin{frame}[c]
\frametitle{¿Por Qué Python en IA?}
\begin{itemize}
    \item Python es un lenguage de programacion bastante flexible.
    \item Permite procesar grandes cantidades de datos de manera rapida.
    \item Permite modificar y visualizar datos de manera eficiente.
    \item Tiene la mayor parte de librerias de codigo abierto de IA y Machine Learning.
    
    %\anotar{La solución óptima del problema general depende de la solución óptima de cada subproblema.}
    %\item Es una optimización sobre las llamadas recursivas simples.
    %\anotar{En una recursión simple no se aprovechan los datos ya calculados.}
\end{itemize}
\end{frame}


\begin{frame}[fragile]
\frametitle{Basicos de Python: Todo es un objeto}

\begin{itemize}
    \item Todos los elementos de Python son \textbf{objetos}.

    \item Por ejemplo, cada numero, texto (string), tabla, etc es un objeto.
    
    \item  Pueden pensar un objeto como una caja con atributos y funcionalidades. 

    \item Cada objeto tiene asociado un \textbf{tipo}. Por ejemplo:

    \begin{itemize}
        \item Todos los textos son del tipo \textbf{\emph{string}}.
        \item Todos los numeros enteros son del tipo \textbf{\emph{int}} ($\{1,-3,4,-66,...\}$).
    \end{itemize}

\end{itemize}

\textbf{Ejemplo}: El objeto \textbf{list} permite almacenar objetos de todo tipo.

\begin{minted}{python}
>>> list = [1,2,3,"cuatro"]
>>> list.reversed() # funcionalidad
>>> list[2]         # funcionalidad
\end{minted}    

\end{frame}

\begin{frame}[fragile]
\frametitle{Tipos de Datos Nativos o Escalares}
Python tiene un pequeño conjunto de datos nativos que también son llamados \emph{escalares}. Estos son datos base que se utilizan en cualquier aplicación.

\begin{table}[]
    \centering
    \begin{tabular}{c|c|c}
        Escalar & Uso & Ejemplo  \\
        \hline
        
        None & Valor Nulo & \code{a = None} \\
        str & Datos de tipo string o texto & a = "Hello World" \\
        %bytes & & \\
        float & Número de punto flotante de doble precisión &  a = 3.1415 \\
        bool & Un valor que puede ser True o False & a = False \\
        int & Entero de precisión arbitraria & a = 33 \\
    \end{tabular}
    %\caption{Caption}
    \label{tab:placeholder}
\end{table}

Vamos a ver mas ejemplos en código: \href{https://colab.research.google.com/drive/13i1U6K5iS2wT86hkLhd9T2-UOo8dY1Gw?usp=sharing}{Python Clase 1}. 


\end{frame}




\begin{frame}[fragile]
\frametitle{Extras}
%imports
\begin{itemize}
    \item \textbf{Imports}: Sirven para invocar funcionalidades de un paquete.
\begin{minted}{python}
import numpy as np
a = np.array( (2,2) )
import matplotlib.pyplot as plt
plt.plot([1,2,3,4,5])
\end{minted}

    \item \textbf{Comentarios} sirven para añadir información adicional al código. Pueden mejorar la legibilidad si son usados correctamente.

\begin{minted}{python}
import numpy as np #importando libreria numpy en python
b = 22 #definiendo una nueva variable b
\end{minted}

    
    \item Una \textbf{Lista} es un objeto que para guarda elementos de diferentes tipos.
\begin{minted}{python}
a = [1,2,3,"cuatro",True]
\end{minted}    


\end{itemize}





\end{frame}



\begin{frame}[fragile]
\frametitle{Condicionales e Indentación}

\begin{itemize}
    \item Python tiene condicionales lógicos para \textbf{controlar el flujo de ejecución del código}.
    \item Para entender los condicionales necesitamos introducir el concepto de \textbf{indentación}.
    \item Una indentación consiste en una cantidad (4) de espacios generados con el comando TAB. 
    \item La indentación después de escribir condiciones lógicas, iteradores seguidos de un símbolo ":".

\end{itemize}

\textbf{Ejemplo}: Simularemos una condición "si ocurre esto entonces"
\begin{minted}{python}
>>> num = 3
>>> if num > 0:
...     print("positivo!") # esta linea tiene una indentacion
\end{minted}

Out: $\code{\text{positivo!}}$

\end{frame}





\begin{frame}[fragile]
\frametitle{Condicionales}

\begin{itemize}
    \item \textbf{If}: Simula la condicion "si ocurre esto entonces"
    \begin{minted}{python}
num = 3
if num > 0:
    print("positivo!")
\end{minted}
    \item elif Es una condicion alternativa que puede ocurrir despues de un If  
\begin{minted}{python}
num = -2
if num > 0:
    print("num es positivo!")
elif num < 0:
    print("num es negativo!")
\end{minted}

\end{itemize}
\end{frame}

\begin{frame}[fragile]
\frametitle{Condicionales}

\begin{itemize}
    \item \textbf{else} Ejecuta un bloque de código cuando todas las condiciones previas son falsas
\begin{minted}{python}
num = 0
if num > 0:
    print("num es positivo!")
elif num < 0:
    print("num es negativo!")
else:
    print("num es cero")
\end{minted}

\end{itemize}


\textbf{Vamos a ver mas ejemplos en código}: \href{https://colab.research.google.com/drive/13i1U6K5iS2wT86hkLhd9T2-UOo8dY1Gw?usp=sharing}{Python Clase 1}. 


\end{frame}






\begin{frame}{Referencias:}
    \begin{itemize}
        \item Wes McKinney. (2022). Python for Data Analysis. Third Edition.
    \end{itemize}
\end{frame}


\end{document}