\documentclass[11pt,aspectratio=169,handout]{beamer}
\usepackage{borelian}

\begin{document}
    \classtitle{1}{Introducción a Python para Ciencia de Datos}{5 de enero de 2026}

    \begin{frame}{¿Por qué Python?}
        \begin{itemize}
            \item Python es un lenguaje de programación bastante flexible.
            \pause
            \item Permite procesar grandes cantidades de datos de manera rápida.
            \pause
            \item Permite modificar y visualizar datos de manera eficiente.
            \pause
            \item Tiene la mayor parte de librerías de código abierto de IA y Machine Learning.
        \end{itemize}
    \end{frame}

    \begin{frame}[fragile]{Básicos de Python: todo es un objeto}
        \begin{itemize}
            \item Todos los elementos de Python son objetos.        
            \begin{itemize}
                \item Por ejemplo, cada número, texto (\textit{string}), tabla, etc.
            \end{itemize}
            \pause
            \item Pueden pensar un objeto como una caja con atributos y funcionalidades.
            \pause
            \item Cada objeto tiene asociado un tipo de dato. Por ejemplo:
            \begin{itemize}
                \item Todos los textos son del tipo \emph{string}.
                \item Todos los números enteros son del tipo \emph{int} ($\{1, -3, 4, -66, \dots\}$).
            \end{itemize}
        \end{itemize}

        \pause
        \begin{exampleblock}{Ejemplo}
            El objeto \emph{list} permite almacenar objetos de todo tipo.

            \begin{minted}{python}
                >>> lista = [1, 2, 3, "cuatro"]
                >>> lista.reversed()
                >>> lista[2]
            \end{minted}
        \end{exampleblock}
    \end{frame}

    \begin{frame}[fragile]{Tipos de datos nativos o escalares}
        Python tiene un pequeño conjunto de datos nativos que también son llamados \emph{escalares}. Estos son datos base que se utilizan en cualquier aplicación.

        \pause
        \begin{table}[H]
            \centering
            \begin{tabular}{c|l|l}
                Escalar & Descripción & Representación \\
                \hline  
                \mintinline{python}{None} & Valor nulo & \mintinline{python}{None} \\
                \mintinline{python}{str} & Datos de tipo \emph{string} o texto & \mintinline{python}{"¡Hola, mundo!"} \\
                \mintinline{python}{float} & Número de punto flotante de doble precisión & \mintinline{python}{3.1415} \\
                \mintinline{python}{bool} & Un valor que puede ser \emph{True} o \emph{False} & \mintinline{python}{False} \\
                \mintinline{python}{int} & Entero de precisión arbitraria & \mintinline{python}{33} \\
            \end{tabular}
            \label{tab:placeholder}
        \end{table}

        \pause
        Vamos a ver más ejemplos en código: \href{https://colab.research.google.com/drive/13i1U6K5iS2wT86hkLhd9T2-UOo8dY1Gw?usp=sharing}{Material complementario}. 
    \end{frame}

    \begin{frame}[fragile]{Extras}
        \begin{itemize}
            \item Los \emph{imports} sirven para invocar funcionalidades de un paquete.
            \begin{minted}{python}
                >>> import numpy as np
                >>> np.array((2, 2))
            \end{minted}

            \pause
            \item Los comentarios sirven para añadir información adicional al código. Pueden mejorar la legibilidad si son usados correctamente. Hay estándares que recomendamos seguir, como PEP 257 \citep{pep257}.
            \begin{minted}{python}
                >>> import matplotlib.pyplot as plt
                >>> y = [3, 2, 1]
                >>> plt.plot(y) # Recta que pasa por (0, 3), (1, 2) y (2, 1).
                >>> plt.show() # Muestra el gráfico en pantalla.
            \end{minted} 
        \end{itemize}
    \end{frame}

    \begin{frame}[fragile]{Condicionales e indentación}
        \begin{itemize}
            \item Python tiene condicionales lógicos para controlar el flujo de ejecución del código.
            \pause
            \item Para entender los condicionales, necesitamos introducir el concepto de indentación.
            \pause
            \item Una indentación consiste en una cantidad de espacios generados con la tecla \tikzexternaldisable\keys{Tab}\tikzexternalenable. 
            \pause
            \item Al final de la línea anterior a una indentación, se deben colocar dos puntos \mintinline{python}{:} como marca indicadora.
        \end{itemize}

        \pause
        \begin{exampleblock}{Simulación de una condición}
            \begin{minted}{python}
                >>> num = 3
                >>> if num > 0:
                ...     print("¡Positivo!") # Con indentación.
                "¡Positivo!"
            \end{minted}
        \end{exampleblock}
    \end{frame}

    \begin{frame}[fragile]{Condicionales}
        \begin{itemize}
            \item La palabra reservada \mintinline{python}{if} simula la condición ``si ocurre $X$ entonces $Y$'':
            \begin{minted}{python}
                num = 3
                if num > 0:
                    print("¡Positivo!")
            \end{minted}

            \pause
            \item La palabra reservada \mintinline{python}{elif} es una condición alternativa que puede ocurrir después de un \mintinline{python}{if}:  
            \begin{minted}{python}
                num = -2
                if num > 0:
                    print("¡Positivo!")
                elif num < 0:
                    print("¡Negativo!")
            \end{minted}
        \end{itemize}
    \end{frame}

    \begin{frame}[fragile]{Condicionales}
        \begin{itemize}
            \item La palabra reservada \mintinline{python}{else} ejecuta un bloque de código cuando todas las condiciones previas son falsas:
            \begin{minted}{python}
                num = 0
                if num > 0:
                    print("¡Positivo!")
                elif num < 0:
                    print("¡Negativo!")
                else:
                    print("¡Cero!")
            \end{minted}
        \end{itemize}

        \pause
        Vamos a ver más ejemplos en código: \href{https://colab.research.google.com/drive/13i1U6K5iS2wT86hkLhd9T2-UOo8dY1Gw?usp=sharing}{Material complementario}. 
    \end{frame}

    \begin{frame}{Referencias}
        \nocite{mckinney2022}
        \bibliographystyle{plainnat}
        \bibliography{../../references}
    \end{frame}
\end{document}