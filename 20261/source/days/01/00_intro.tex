\documentclass[handout,aspectratio=169]{beamer}
\usepackage{borelian}

\begin{document}
    \classtitle{0}{Introducción}{5 de enero de 2026}

    \begin{frame}{Motivación}
    Para partir el curso, generemos discusión.
    \begin{itemize}
        \item En una palabra... ¿qué entienden ustedes hoy como Inteligencia Artificial?
    \end{itemize}

    \begin{figure}[H]
        \centering
        \includegraphics[width=0.3\linewidth]{days/01/images/mentimeter_qr_code.png}
    \end{figure}
    \end{frame}

    \begin{frame}{¿De qué se trata el curso?}
        \textbf{Vamos a explicar varias cosas...}
        
        \tikzexternaldisable
        \begin{tikzpicture}[remember picture, overlay]
            \node[inner sep=0pt] at (current page.center) [xshift=4cm, yshift=3cm] {
                \includegraphics[width=0.45\paperwidth]{days/01/images/data_evolution.jpeg}
            };

            \node[inner sep=0pt] at (current page.center) [xshift=2.5cm, yshift=-2.25cm] {
                \includegraphics[width=0.35\paperwidth]{days/01/images/statistics_maybe.png}
            };

            \node[inner sep=0pt] at (current page.center) [xshift=-3.5cm, yshift=2.25cm] {
                \includegraphics[width=0.35\paperwidth]{days/01/images/strava_military.png}
            };

            \node[inner sep=0pt] at (current page.center) [xshift=2cm, yshift=.5cm] {
                \includegraphics[width=0.15\paperwidth]{days/01/images/kanizsa_triangle.svg.png}
            };

            \node[inner sep=0pt] at (current page.center) [xshift=5.5cm, yshift=.5cm] {
                \includegraphics[width=0.2\paperwidth]{days/01/images/mond-vergleich.svg.png}
            };

            \node[inner sep=0pt] at (current page.center) [xshift=-4cm, yshift=-2cm] {
                \includegraphics[width=0.35\paperwidth]{days/01/images/divorce_rate.png}
            };
        \end{tikzpicture}
        \tikzexternalenable
    \end{frame}

    \begin{frame}{¿De qué se trata el curso?}
        Este es un curso más \textbf{práctico} que teórico. Nos enfocaremos en las aplicaciones de modelos de Inteligencia Artificial, profundizando lo justo y necesario en el ``por qué'' (para ahondar más, necesitan más herramientas matemáticas).
    \end{frame}

    \begin{frame}{El impacto de la ciencia de datos}
        \begin{columns}
            \begin{column}{0.4\linewidth}
                Veamos cómo anda un minuto de nuestro día.
            \end{column}
            
            \begin{column}{0.65\linewidth}
                \tikzexternaldisable
                \begin{tikzpicture}[remember picture, overlay]
                    \node[anchor=east, inner sep=0pt] at (current page.east) [xshift=-1cm] {
                        \includegraphics[width=0.5\paperwidth, keepaspectratio]{days/01/images/day_minute.png} 
                    };
                \end{tikzpicture}
                \tikzexternalenable           
            \end{column}
        \end{columns}
    \end{frame}

    \begin{frame}{¿De qué se trata el curso?}
        En problemas de Ciencias de la Computación clasicos, resolvemos problemas considerando un input y un programa o algoritmo que genera un output.

        \begin{figure}[H]
            \centering
            \includegraphics[width=\linewidth]{days/01/images/cs_ml1.png.png}
            \caption{Distribución por sexo biológico.}
        \end{figure}

    \end{frame}

    
    \begin{frame}{¿De qué se trata el curso?}
        Que pasa si nos enfrentamos a un problema desconocido y nuevo, donde no conocemos un algoritmo perfecto que resuelve el problema ?

        \begin{figure}[H]
            \centering
            \includegraphics[width=0.7\linewidth]{days/01/images/cs_ml2.png.png}
            \caption{Distribución por sexo biológico.}
        \end{figure}

    \end{frame}


    \begin{frame}{¿De qué se trata el curso?}
        En estos escenarios podemos usar Machine Learning !

        \begin{figure}[H]
            \centering
            \includegraphics[width=0.7\linewidth]{days/01/images/cs_ml3.png.png}
            %\caption{Distribución por sexo biológico.}
        \end{figure}

    \end{frame}

        \begin{frame}{¿De qué se trata el curso?}
        En este curso estudiaremos este tipo de problemas.

        \begin{figure}[H]
            \centering
            \includegraphics[width=\linewidth]{days/01/images/cs_ml4.png.png}
        \end{figure}

    \end{frame}

    \begin{frame}{Caracterización de las y los estudiantes}
        Este semestre se matricularon 38 estudiantes, casi el doble del total de la versión anterior.
        \begin{columns}[c]
            \begin{column}{0.48\linewidth}
                \begin{figure}[H]
                    \centering
                    \includegraphics[width=\linewidth]{scripts/00/out/sex_distribution.pdf}
                    \caption{Distribución por sexo biológico.}
                    \label{fig:biologic_sex_distribution}
                \end{figure}
            \end{column}
            \begin{column}{0.48\linewidth}
                \begin{figure}[H]
                    \centering
                    \includegraphics[width=\linewidth]{scripts/00/out/course_distribution.pdf}
                    \caption{Distribución por curso escolar.}
                    \label{fig:course_distribution}
                \end{figure}
            \end{column}
        \end{columns}
    \end{frame}

    \begin{frame}{Caracterización del equipo docente}
        \begin{columns}
            \begin{column}{0.45\linewidth}
                \begin{figure}[H]
                    \centering
                    \includegraphics[width=0.7\linewidth]{days/01/images/maxfloresv.jpg}
                \end{figure}
                \begin{center}
                {\color{edvdarkgreen75} \textbf{Máximo Flores Valenzuela}}
                
                DCC y M. Sc. (c), Ciencia de Datos de la U. Chile.
                \end{center}
            \end{column}
            \begin{column}{0.45\linewidth}
                \begin{figure}[H]
                    \centering
                    \includegraphics[width=0.7\linewidth]{days/01/images/hectorjimenez12.png}
                \end{figure}
                \begin{center}            
                {\color{edvdarkgreen75} \textbf{Héctor Jiménez Orellana}}

                Ingeniero Civil y M. Sc, Ciencias de la Computación de la U. Chile.
                \end{center}
                
            \end{column}
        \end{columns}
    \end{frame}

    \begin{frame}{Evaluaciones}
        Las evaluaciones del curso consisten en dos tareas, con fechas de publicación y entrega ya definidas:
        \begin{itemize}
            \item Tarea 1: Análisis exploratorio de datos.
            \pause
            \item Tarea 2: Regresión y aprendizaje supervisado.
        \end{itemize}
        \pause
        \begin{table}[H]
            \centering
            \begin{tabular}{ccc}
                & Fecha de publicación & Fecha de entrega \\ \hline
                Tarea 1 & jueves 8 & lunes 12 \\
                Tarea 2 & martes 13 & viernes 16
            \end{tabular}
        \end{table}
        \pause
        Todas las tareas, \textbf{sin excepción}, se publicarán a las 8:00 y deberán ser entregadas a las 13:00 del día que corresponda. No habrá extensión de plazos.
    \end{frame}

    \begin{frame}{Evaluaciones}
        Con ambas tareas, se calcula el promedio simple entre sus notas. Esto entrega una nota que llamaremos ``Promedio de tareas'':
        \[
            \overline{X}_{\mathrm{T}} = 50~\% \cdot \mathrm{T1} + 50~\% \cdot \mathrm{T2}
        \]
        \pause
        También habrá una nota de ``Coevaluación'' ($\mathrm{C}$). Esta la asignarán ustedes a cada integrante de su equipo el último día del curso.
        
        \pause
        La nota final ($\mathrm{NF}$) del curso se calcula, si tanto el promedio de tareas como la coevaluación son $\ge 4.0$:
        \[
            \mathrm{NF} = 80~\% \cdot \overline{X}_{\mathrm{T}} + 20~\% \cdot \mathrm{C}
        \]
        En el caso contrario, el/la estudiante reprueba el curso.
    \end{frame}

    \begin{frame}{Reglas del juego}
        Para garantizar un correcto desarrollo del curso, es necesario imponer las siguientes reglas:

        \begin{itemize}
            \item Las tareas se desarrollan en equipos de 3 a 4 personas. Ni más, ni menos. Está \underline{\textbf{estrictamente prohibido}} incurrir en plagio con otros equipos. Sin embargo, puede discutir ideas en clases o fuera de ellas, siguiendo la \textit{Whiteboard Policy}.
            \pause
            \item En las evaluaciones, pueden apoyarse de modelos grandes de lenguaje, p. ej., ChatGPT, pero como se espera que logren objetivos de aprendizaje, está \textbf{\underline{estrictamente prohibido}} copiar y pegar cualquier contenido (texto, código, etc.) sin entender lo que se está realizando.
            \pause
            \item Si extrae información de una fuente externa, especialmente cuando es algo que no se ha visto en el curso, debe citar debidamente el origen del contenido.
            \pause
        \end{itemize}

        Como consecuencia del incumplimiento de alguna de estas reglas, se evaluará con la nota mínima a las y los involucrados.
    \end{frame}

    \begin{frame}{Bonificación por participación}
        La asistencia a las cátedras \textbf{no es obligatoria}; sin embargo, como el curso avanza a un ritmo acelerado, es muy recomendado asistir para sacarle el máximo provecho.
        
        En esta versión del curso, implementaremos una trivia después de cada clase, usando la plataforma \href{https://www.mentimeter.com/es-ES}{Mentimeter}. Esta trivia tiene un podio, y quienes logren quedar en él, obtendrán las siguientes bonificaciones:
        \begin{itemize}
            \item 1.\textsuperscript{er} lugar: $\mathbf{3}$ \textbf{décimas}, usables en cualquiera de las 2 tareas.
            \item 2.\textsuperscript{do} lugar: $\mathbf{2}$ \textbf{décimas}, usables sólo en una de las dos tareas.
            \item 3.\textsuperscript{er} lugar: $\mathbf{1}$ \textbf{décima}, usable sólo en una de las dos tareas.
        \end{itemize}

        \begin{figure}[H]
            \centering
            \includegraphics[width=0.5\linewidth]{days/01/images/mentimeter.pdf}
            \caption{Imagotipo de Mentimeter.}
            \label{fig:mentimeter}
        \end{figure}
    \end{frame}

    \begin{frame}{Canales de comunicación}
        Los canales de comunicación oficiales serán \href{https://www.u-cursos.cl/escverano/2026/3/FM849/1/historial/}{U-Cursos} y \href{https://github.com/maxfloresv/FM849}{GitHub}. Allí, se encontrará todo el material relacionado al curso y nos podrán contactar. 

        \pause
        Si pierden acceso a alguna plataforma, les dejamos nuestros correos en esta \textit{slide}:
        \begin{center}
            \texttt{mflores@dcc.uchile.cl}~~~\texttt{hector.jimenezor@gmail.com}
        \end{center}

        \pause
        \textbf{\color{edvred50} Regla de oro}: escríbannos en horarios razonables (9:00 -- 19:00). Pueden programar correos si están trabajando fuera de estos horarios. Asegúrense de que su duda no esté respondida.
    \end{frame}
\end{document}