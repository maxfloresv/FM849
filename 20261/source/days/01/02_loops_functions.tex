\documentclass[11pt,t,aspectratio=169]{beamer}
\usepackage{borelian}
\usepackage{minted}

\begin{document}
\classtitle{2}{Ciclos y funciones en Python.}

%%%%%%%
\begin{frame}{Contenidos de hoy}
    


    \begin{itemize}%[<+->]
        \item Ciclos en Python.

        \item Funciones en Python.
        
        \item Clases en Python.
        
    \end{itemize}
    
\end{frame}

\begin{frame}[c]
\frametitle{Condicionales, Iteradores y Otros}
Python tiene condicionales logicos, iteradores y otros operadores de flujo para \textbf{controlar el orden de ejecución de instrucciones}.

\begin{table}[]
    \centering
    \begin{tabular}{c|c|c}
        Tipo de Operador & Uso & Operadores  \\
        \hline
        
        Condicionales & Ejecutar código según condiciones & \code{if, elif, else} \\
        Loops & Recorrer conjuntos de datos & \code{for, while} \\
        Otros & Funcionalidades utiles & \code{pass,break,range} \\
    \end{tabular}
    %\caption{Caption}
    \label{tab:placeholder}
\end{table}

Vamos a ver mas ejemplos en código: \href{https://colab.research.google.com/drive/13i1U6K5iS2wT86hkLhd9T2-UOo8dY1Gw?usp=sharing}{Python Clase 1}. 


\end{frame}



%%%%%%%%%%%%%%%%%%%%%%%%%%%%%%%%%%%%%%%%%%%%%%%%%%%%%%%%%%%%

\begin{frame}[fragile]
\frametitle{Iteradores: For}

El ciclo \textcolor{purple}{for} se utiliza para recorrer elementos en una secuencia (tuplas, lista, vectores, etc).

\vspace{0.5mm}
\textbf{Sintaxis general:}
\vspace{-1mm}
\begin{minted}{python}
>>> for variable in secuencia:
...     # código a ejecutar
\end{minted}

\vspace{5mm}

\begin{columns}
\column{0.5\textwidth}
\textbf{Ejemplo:}
\begin{minted}{python}
for i in range(5):
    print(i)
\end{minted}

\column{0.5\textwidth}
\textbf{Salida:}
\begin{minted}{python}
0
1
2
3
4
\end{minted}

\end{columns}

\end{frame}

\begin{frame}[fragile]
\frametitle{Iteradores: While}
El ciclo while ejecuta un bloque de código mientras una condición sea cierta.

\textbf{Sintaxis general:}
\vspace{-1mm}
\begin{minted}{python}
>>> while condition:
...     # código a ejecutar mientras condition sea True
\end{minted}

\vspace{5mm}

\begin{columns}
\column{0.5\textwidth}
\textbf{Ejemplo:}
\begin{minted}{python}
k=0
while k<5:
    print(i)
    k += 1 # k = k + 1
\end{minted}

\column{0.5\textwidth}
\textbf{Salida:}
\begin{minted}{python}
0
1
2
3
4
\end{minted}

\end{columns}


\end{frame}

%%%%%%%%%%%%%%%%%%%%%%%%%%%%%%%%%%%%%%%%%%%%%%%%%%%%%%

\begin{frame}[fragile]
\frametitle{Loop Control Statements: Continue}
La instrucción continue genera un salto a la siguiente iteración del ciclo, ignorando el código siguiente.


\textbf{Sintaxis general:}
\vspace{-1mm}
\begin{minted}{python}
>>> while condition1:
...     if condition2:
...        continue
...     # código no ejecutado si condition == True
\end{minted}

\vspace{5mm}

\begin{columns}
\column{0.5\textwidth}
\textbf{Ejemplo:}
\begin{minted}{python}
for i in range(5):
    if i == 2:
        continue
    print(i)
\end{minted}

\column{0.5\textwidth}
\textbf{Salida:}
\begin{minted}{python}
0
1
3
4
\end{minted}

\end{columns}

\end{frame}

\begin{frame}[fragile]
\frametitle{Loop Control Statements: Break}
La instrucción break termina completamente el loop.

\textbf{Sintaxis general:}
\vspace{-1mm}
\begin{minted}{python}
>>> while condition1:
...     if condition2:
...        break
... # si condition2 es True, break se activa y continuamos aca
\end{minted}

\vspace{5mm}

\begin{columns}
\column{0.5\textwidth}
\textbf{Ejemplo:}
\begin{minted}{python}
for i in range(5):
    if i == 3:
        break
    print(i)
\end{minted}

\column{0.5\textwidth}
\textbf{Salida:}
\begin{minted}{python}
0
1
2
\end{minted}
\end{columns}
\end{frame}

\begin{frame}[fragile]
\frametitle{Loop Control Statements: Pass}
La instrucción pass es un comando que indica no hacer nada.

\textbf{Ejemplo:}
\vspace{-1mm}
\begin{minted}{python}
>>> if x < 0:
...     print("negativo!")
... elif x == 0:
...     # TODO: poner algo aquí
...     pass
... else:
...     print("positivo!")
\end{minted}


\end{frame}


\begin{frame}[fragile]
\frametitle{Funciones}
Una función es un bloque de código reutilizable. Algunas características:

\begin{itemize}
    \item Tiene un objetivo especifico.
    \item Usualmente recibe \textbf{parámetros}, también conocidos como inputs.
    \item Comúnmente entregara un \textbf{resultado}.
\end{itemize}

\textbf{Ejemplo:}
\vspace{-1mm}
\begin{minted}{python}
>>> # funcion que retorna la suma de dos numeros
>>> def my_function(x, y):
...    return x + y
\end{minted}


\end{frame}


\begin{frame}[fragile]
\frametitle{Funciones}



\begin{itemize}
    \item Las funciones se definen usando la palabra clave \textbf{def}.
    
    \item Luego de \textbf{def} se escribe el nombre de la funcion y entre parentesis, los parametros que esta recibe.

    \item En las siguientes lineas de codigo se escriben todas las funcionalidades de la funcion.

    \item Cuando se alcanza una linea que tenga el comando \textbf{return}, entonces la funcion devuelve el valor despues de return en el contexto en el cual fue llamada.
\end{itemize}


\begin{columns}
    \column{0.5\textwidth}
\textbf{Sintaxis general:}
\begin{minted}{python}
def nombre_funcion(parametros):
    # cuerpo de la función
    return resultado
\end{minted}

    \column{0.5\textwidth}
\textbf{Ejemplo:}
\begin{minted}{python}
def suma(a, b):
    return a + b
resultado = suma(3, 5)
print(resultado)
\end{minted}
\end{columns}





    
\end{frame}


\begin{frame}[fragile]
\frametitle{Funciones}
\begin{itemize}
    \item Las funciones ayudan a organizar y reutilizar codigo en Python.
    
    \item Como regla general, si sabemos que una parte del código sera usada múltiples veces, podemos crear una función que simule esto.
\end{itemize}
\begin{columns}

    \column{0.5\textwidth}
    \textbf{Sin funciones}:

    \begin{minted}{python}
>>> l = [2,3,4,5,2]
>>> max_valor = - np.inf
>>> for i in range(len(l)):
...     if l[i] > max_valor:
...         max_valor = l[i]
>>> l = [255,313,42,53,20]:
>>> max_valor = - np.inf
>>> for i in range(len(l)):
...     if l[i] > max_valor:
...         max_valor = l[i]
\end{minted}

    
    
    \column{0.5\textwidth}
    \textbf{Con funciones}:
    \begin{minted}{python}
>>> def max(lista):
...     max_valor = - np.inf
...     for i in range(len(l)):
...         if l[i] > max_valor:
...             max_valor = l[i]
...     return max_valor
>>> l = [2,3,4,5,2]
>>> max_valor = max(l)
>>> l = [255,313,42,53,20]
>>> max_valor = max(l)

\end{minted}
    
\end{columns}

    
\end{frame}


%%%%%%%%%%%%%%%%%%%%%%%%%%%%%%%%%%%%%%%%%%%%%%%%%%%

\begin{frame}[fragile]
\frametitle{Clases}
\begin{itemize}
    \item Las clases proporcionan un medio para agrupar datos y funcionalidades.

    \item Cuando creamos una clase nueva podemos generar objetos con nuevas características.

    \item Las clases tienen dos cosas fundamentales: \textbf{atributos} y \textbf{metodos}.
\end{itemize}


\begin{minted}{python}
>>> class Complex:
...     def __init__(self, realpart, imagpart):
...         self.r = realpart
...         self.i = imagpart
>>> x = Complex(3.0, -4.5)
>>> x.r, x.i

(3.0, -4.5)
\end{minted}

\end{frame}

%%%%%%%%%%%%%%%%%%%%%%%%%%%%%%%%%%%%%%%%%%%%%%%%%%%






\begin{frame}{Referencias:}
    \begin{itemize}
        \item Wes McKinney. (2022). Python for Data Analysis. Third Edition.
        \item \url{https://www.geeksforgeeks.org/python/loops-in-python/}
    \end{itemize}
\end{frame}


\end{document}