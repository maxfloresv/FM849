\documentclass[11pt,aspectratio=169]{beamer}
\usepackage{borelian}

\begin{document}
    \classtitle{2}{Ciclos y funciones en Python}{5 de enero de 2026}

    \begin{frame}{Condicionales y ciclos}
        Python tiene condicionales lógicos, ciclos y otros operadores de flujo para controlar el orden de ejecución de instrucciones.

        \begin{table}[H]
            \centering
            \begin{tabular}{c|l|l}
                Tipo de operador & Uso & Operadores  \\
                \hline
                Condicionales & Ejecutar código según condiciones & \mintinline{python}{if, elif, else} \\
                Ciclos (\textit{loops}) & Recorrer conjuntos de datos & \mintinline{python}{for, while} \\
                Controladores de ciclo & Funcionalidades útiles & \mintinline{python}{continue, break, pass}
            \end{tabular}
        \end{table}

        Vamos a ver más ejemplos en código: \href{https://colab.research.google.com/drive/13i1U6K5iS2wT86hkLhd9T2-UOo8dY1Gw?usp=sharing}{Material complementario}.
    \end{frame}

    \begin{frame}[fragile, t]{Ciclos: \mintinline{python}{for}}
        El ciclo \mintinline{python}{for} se utiliza para recorrer elementos en un iterable (p. ej., tuplas, listas, vectores, etc.).

        \begin{block}{Sintaxis general}
            \begin{minted}{python}
                >>> for variable in iterable:
                ...     # Código a ejecutar en cada iteración.
            \end{minted}    
        \end{block}

        \vspace{-\baselineskip}
        \begin{columns}[t]
            \begin{column}{0.45\linewidth}
                \begin{exampleblock}{Ejemplo}
                    \begin{minted}{python}
                        for i in range(5):
                            print(i)
                    \end{minted}
                \end{exampleblock}
            \end{column}
            \hfill
            \begin{column}{0.45\linewidth}
                \begin{exampleblock}{Salida}
                    \begin{minted}{python}
                        0
                        1
                        2
                        3
                        4
                    \end{minted}                    
                \end{exampleblock}
            \end{column}
        \end{columns}
    \end{frame}

    \begin{frame}[fragile]{Ciclos: \mintinline{python}{while}}
        El ciclo \mintinline{python}{while} ejecuta un bloque de código mientras una condición sea cierta.

        \begin{block}{Sintaxis general}
            \begin{minted}{python}
                >>> while condition:
                ...     # Código a ejecutar mientras condition == True.
            \end{minted}
        \end{block}

        \vspace{-\baselineskip}
        \begin{columns}[t]
            \begin{column}{0.45\linewidth}
                \begin{exampleblock}{Ejemplo}
                    \begin{minted}{python}
                        k = 0
                        while k < 5:
                            print(i)
                            k += 1 # k = k + 1.
                    \end{minted}                    
                \end{exampleblock}
            \end{column}
            \hfill
            \begin{column}{0.45\linewidth}
                \begin{exampleblock}{Salida}
                    \begin{minted}{python}
                        0
                        1
                        2
                        3
                        4
                    \end{minted}
                \end{exampleblock}                
            \end{column}
        \end{columns}
    \end{frame}

    \begin{frame}[fragile]{Control de ciclo: \mintinline{python}{continue}}
        La instrucción \mintinline{python}{continue} genera un salto a la siguiente iteración del ciclo, ignorando el código que sigue.

        \begin{block}{Sintaxis general}
            \begin{minted}{python}
                >>> while condition1:
                ...     if condition2:
                ...        continue
                ...     # Código no ejecutado si condition2 == True.
            \end{minted}
        \end{block}
        
        \vspace{-\baselineskip}
        \begin{columns}
            \begin{column}{0.45\linewidth}
                \begin{exampleblock}{Ejemplo}
                    \begin{minted}{python}
                        for i in range(5):
                            if i == 2:
                                continue
                            print(i)
                    \end{minted}
                \end{exampleblock}
            \end{column}
            \hfill
            \begin{column}{0.45\linewidth}
                \begin{exampleblock}{Salida}
                    \begin{minted}{python}
                        0
                        1
                        3
                        4
                    \end{minted}
                \end{exampleblock}
            \end{column}
        \end{columns}
    \end{frame}

    \begin{frame}[fragile]{Control de ciclo: \mintinline{python}{break}}
        La instrucción \mintinline{python}{break} termina completamente el ciclo.

        \begin{block}{Sintaxis general}
            \begin{minted}{python}
                >>> while condition1:
                ...     if condition2:
                ...        break
                ... # Si condition2 == True, break se activa y continuamos acá.
            \end{minted}            
        \end{block}

        \vspace{-\baselineskip}
        \begin{columns}
            \begin{column}{0.45\linewidth}
                \begin{exampleblock}{Ejemplo}
                    \begin{minted}{python}
                    for i in range(5):
                        if i == 3:
                            break
                        print(i)
                    \end{minted}              
                \end{exampleblock}
            \end{column}
            \hfill
            \begin{column}{0.45\linewidth}
                \begin{exampleblock}{Salida}
                    \begin{minted}{python}
                        0
                        1
                        2
                    \end{minted}                    
                \end{exampleblock}                
            \end{column}
        \end{columns}
    \end{frame}

    \begin{frame}[fragile]{Control de ciclo: \mintinline{python}{pass}}
        La instrucción \mintinline{python}{pass} es un comando que indica no hacer nada.

        \begin{exampleblock}{Ejemplo (supondremos que \mintinline{python}{x} ya está definido)}
            \begin{minted}{python}
                >>> if x < 0:
                ...     print("¡Negativo!")
                ... elif x == 0:
                ...     # TODO: agregar el caso x == 0.
                ...     pass
                ... else:
                ...     print("¡Positivo!")
            \end{minted}
        \end{exampleblock}
    \end{frame}

    \begin{frame}[fragile]{Funciones}
        Una función es un bloque de código ``reutilizable'' (cuidado con este concepto, porque existen las \href{https://ellibrodepython.com/lambda-python}{funciones anónimas}). Algunas características:

        \begin{itemize}
            \item Tiene un objetivo especifico.
            \item Usualmente, recibe parámetros, también conocidos como \emph{inputs} en la jerga de programación.
            \item Comúnmente, entregará un resultado.
        \end{itemize}

        \begin{exampleblock}{Ejemplo}
            \begin{minted}{python}
                >>> # Función que retorna la suma de dos números.
                >>> def my_function(x, y):
                ...    return x + y
            \end{minted}
        \end{exampleblock}
    \end{frame}

    \begin{frame}[fragile]{Funciones}
        \begin{itemize}
            \item Las funciones se definen usando la palabra clave \mintinline{python}{def}.
            
            \item Luego de \mintinline{python}{def}, se escribe el nombre de la función y, entre paréntesis, los parámetros que ésta recibe.

            \item En las siguientes líneas de código se escriben todas las funcionalidades de la función.

            \item Cuando se alcanza una línea que tenga el comando \mintinline{python}{return}, la función devuelve el valor después de \mintinline{python}{return} en el contexto en el cual fue llamada.
        \end{itemize}

        \begin{columns}[t]
            \begin{column}{0.45\linewidth}
                \begin{block}{Sintaxis general}
                    \begin{minted}{python}
                        def nombre_funcion(parametros):
                            # Cuerpo de la función.
                    \end{minted}
                \end{block}
            \end{column}
            \hfill
            \begin{column}{0.45\linewidth}
                \begin{exampleblock}{Ejemplo}
                    \begin{minted}{python}
                        def suma(a, b):
                            return a + b
                        resultado = suma(3, 5)
                        print(resultado)
                    \end{minted}
                \end{exampleblock}
            \end{column}
        \end{columns}            
    \end{frame}

    \begin{frame}[fragile]{Funciones}
        \begin{itemize}
            \item Las funciones ayudan a organizar y reutilizar código en Python.
            
            \item Como regla general, si sabemos que una parte del código será usada múltiples veces, podemos crear una función que la abstraiga.
        \end{itemize}

        \vspace{-\baselineskip}
        \begin{columns}[t]
            \begin{column}{0.45\linewidth}
                \begin{block}{Sin funciones}
                    \begin{minted}{python}
                        >>> import numpy as np
                        >>> l = [2, 3, 4, 5, 2]
                        >>> max_valor = -np.inf
                        >>> for i in range(len(l)):
                        ...     if l[i] > max_valor:
                        ...         max_valor = l[i]
                        >>> l = [255, 313, 42, 53, 20]
                        >>> max_valor = -np.inf
                        >>> # Se repite lo mismo aquí...
                    \end{minted}
                \end{block}
            \end{column}
            \hfill
            \begin{column}{0.45\linewidth}
                \begin{block}{Con funciones}
                    \begin{minted}{python}
                        >>> import numpy as np
                        >>> def calc_max(l):
                        ...     max_valor = -np.inf
                        ...     for i in range(len(l)):
                        ...         if l[i] > max_valor:
                        ...             max_valor = l[i]
                        ...     return max_valor
                        >>> l = [2, 3, 4, 5, 2]
                        >>> max_valor = calc_max(l)
                    \end{minted}
                \end{block}
            \end{column}        
        \end{columns}        
    \end{frame}

    \begin{frame}[fragile]{Clases}
        \begin{itemize}
            \item Las clases proporcionan un medio para agrupar datos y funcionalidades.

            \item Cuando creamos una clase nueva, podemos generar objetos con nuevas características.

            \item Las clases tienen dos elementos fundamentales: atributos y métodos.
        \end{itemize}

        \begin{minted}{python}
            >>> class Complex:
            ...     def __init__(self, real_part, imag_part):
            ...         self.r = real_part
            ...         self.i = imag_part
            >>> x = Complex(3.0, -4.5)
            >>> x.r, x.i
            (3.0, -4.5)
        \end{minted}
    \end{frame}

    \begin{frame}{Referencias}
        \nocite{mckinney2022, geeksforgeeks-loops}
        \bibliographystyle{plainnat}
        \bibliography{../../references}
    \end{frame}
\end{document}