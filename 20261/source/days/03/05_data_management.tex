%Herramientas elementales para cargar y manipular datos en Python con Pandas. Formatos principales para guardar datos y sus diferencias.

\documentclass[11pt,t,aspectratio=169]{beamer}
\usepackage{borelian}
\usepackage{multirow}

\usetheme{lehighlight}
\usefonttheme{professionalfonts}

\setbeamersize{
    text margin left=2em,
    text margin right=2em
}

\usepackage{hyperref}
\usepackage{multicol}
\usepackage{minted}
\usepackage{xcolor}  
\usepackage{colortbl}
\usepackage{graphics}
\usepackage{tikz}
\usepackage{ragged2e}
\usepackage{listings}

\usepackage{cmbright}
\usepackage[T1]{fontenc}


\usepackage{listings}

\usepackage{xcolor}

%New colors defined below
\definecolor{codegreen}{rgb}{0,0.6,0}
\definecolor{codegray}{rgb}{0.5,0.5,0.5}
\definecolor{codepurple}{rgb}{0.58,0,0.82}
\definecolor{backcolour}{rgb}{0.95,0.95,0.92}

%Code listing style named "mystyle"
\lstdefinestyle{mystyle}{
  backgroundcolor=\color{backcolour}, commentstyle=\color{codegreen},
  keywordstyle=\color{magenta},
  numberstyle=\tiny\color{codegray},
  stringstyle=\color{codepurple},
  basicstyle=\ttfamily\footnotesize,
  breakatwhitespace=false,         
  breaklines=true,                 
  captionpos=b,                    
  keepspaces=true,                 
  numbers=left,                    
  numbersep=5pt,                  
  showspaces=false,                
  showstringspaces=false,
  showtabs=false,                  
  tabsize=2
}

%"mystyle" code listing set
\lstset{style=mystyle}



\apptocmd{\frame}{}{\justifying}{}

\newcommand{\mpar}{\vspace{3mm}\par}
\newcommand{\complejidad}[1]{\mpar\textsf{\color{blue} Complejidad: $\mathcal{O}(#1)$}.}

\newcommand{\anotar}[1]{\vspace{1mm}{\footnotesize \color{blue} #1}}

\newcommand{\code}[1]{{\color{purple} \textbf{#1}}}

% Delete this, if you do not want the table of contents to pop up at the beginning of each subsection:
\AtBeginSection[]
{
    \begingroup
        \setbeamertemplate{background canvas}[vertical shading]
        \setbeamertemplate{footline}[sectionfootline] 
        \setbeamertemplate{section page}[mysection]
        \frame[c]{
        \sectionpage
        }
    \endgroup
}

\title{\Large Clase 3: Programación Dinámica}
\subtitle{\Large CC4002-CC4006}
\author{}
\institute{Universidad de Chile}
\date{\today}



\begin{document}
\classtitle{7}{Manejo de Datos en Python.}


%%%%%%%
\begin{frame}{Contenidos de hoy}
    
    \begin{itemize}%[<+->]
        \item Introducción a Pandas en Python.
        
        \item Estructuras de Datos en Pandas (Series y DataFrames).
        
        \item Como cargar datos en Pandas.

        \item Como acceder a datos en Pandas.
        
        \item Como guardar datos en Pandas.
    
        

    \end{itemize}
    
\end{frame}


%%%%%%%
\begin{frame}{Paquete Pandas en Python}
    Pandas es una librería muy flexible que permite manejar y visualizar datos. 
\vspace{5mm}
    \begin{itemize}
        \item Contiene estructuras y herramientas para manejar datos de manera conveniente.

        \item Tiene metodos para visualizar datos.
    \end{itemize}

\vspace{5mm}
\textbf{Por que Pandas ?}
    \begin{itemize}
        \item Contiene estructuras y herramientas para manejar datos de manera conveniente.
        \item Es mucho mas flexible que otros programas (ej. Excel), por lo que nos permite hacer cosas mas complicadas y útiles para aplicaciones de IA.
        \item Es capaz de manejar grandes datasets con muchos ejemplos.
    \end{itemize}

\end{frame}



\begin{frame}[fragile]{Empezando con Pandas}

Primero importamos el paquete para hacer uso de todas sus funciones.

\begin{minted}{python}
import pandas # importamos el paquete
data = pandas.read_csv("datos.csv") #usamos las funciones del paquete
\end{minted}

Existe una manera mas conveniente de hacer esto. 
\begin{minted}{python}
import pandas as pd # importamos el paquete
data = pd.read_csv("datos.csv") #usamos las funciones del paquete
\end{minted}
\paragraph{De estas forma nos evitamos usar $\code{pandas}$ en cada llamada a una función. También, comúnmente usaremos la librería $\code{numpy}$ en conjunto con pandas.}

\begin{minted}{python}
import numpy as np
\end{minted}


\end{frame}


%%%%%%%%%%%%%%%%%

\begin{frame}{Estructuras de Datos en Pandas.}
    En Pandas trabajaremos principalmente con dos estructuras de datos.

    \begin{table}[]
        \centering
        \begin{tabular}{c|c|c}
            Dim & Nombre & Descripción \\
            \hline
            1D & \code{pd.Serie} & Arreglo de elementos 1D de un solo tipo \\
            2D & \code{pd.DataFrame} & Arreglo 2D etiquetado mutable con columnas de tipo Serie
        \end{tabular}
        \caption{Tipos de estructuras de datos en Pandas.}
        \label{tab:placeholder}
    \end{table}

    
\end{frame}


\begin{frame}[fragile]{Objetos de tipo Series}
Un objeto de tipo \emph{Series} es un arreglo 1-dimensional que contiene una secuencia de valores del mismo tipo. Este arreglo tiene asociado un arreglo de etiquetas, llamado \emph{index}. 

\begin{minted}{python}
In[1]: obj = pd.Series([4, 7, -5, 3])
In[2]: obj
\end{minted}

\begin{minted}{python}
Out[2]:
0 4
1 7
2 -5
3 3
dtype: int64
\end{minted}
\end{frame}


\begin{frame}[fragile]{Objetos de tipo DataFrame}
Un objeto de tipo \emph{DataFrame} representa una tabla rectangular de datos. Esta contiene una colección de columnas, donde cada una puede ser de un tipo diferente.  

\begin{minted}{python}
In[1]: data = {"state": ["Ohio", "Ohio", "Ohio", "Nevada", "Nevada"],
               "year": [2000, 2001, 2002, 2002, 2003],
               "pop": [1.5, 1.7, 3.6, 2.9, 3.2]}
In[2]: frame = pd.DataFrame(data)
\end{minted}

\begin{minted}{python}
Out[2]:
state year pop
0 Ohio 2000 1.5
1 Ohio 2001 1.7
2 Ohio 2002 3.6
3 Nevada 2002 2.9
4 Nevada 2003 3.2
\end{minted}    
\end{frame}


\begin{frame}[fragile]{Básicos de Pandas}

Supongamos que importamos data y generamos un DataFrame:

\begin{minted}{python}
    df = pd.read_csv("datos.csv") # df es un DataFrame
\end{minted}

Podemos usar:
\begin{itemize}
    \item $\code{df.head()}$ para mirar las primeras filas.
    \item $\code{df.tail()}$ para mirar las ultimas filas.
    \item $\code{df.shape}$ para obtener las dimensiones de la tabla.
    \item $\code{len(df)}$ para obtener el numero de filas en la tabla.
    \item $\code{df.columns}$ para obtener los nombres de las columnas en la tabla.
    \item $\code{df.iloc[ k ]}$ para obtener la fila ubicada en la posición $k$.
    \item $\code{df.loc[index]}$ para obtener la fila asignada al indice $index$.
\end{itemize}

\textbf{Pandas tiene muchas mas funciones! (veremos esto en un ejemplo practico con código)}.index
    
\end{frame}


\begin{frame}{Carga de datos}

\paragraph{En los ejemplos anteriores generamos Series y DataFrames usando datos creados de manera manual. Sin embargo, en proyectos reales debemos cargar datos desde datasets o paginas webs. Esto se puede realizar con las siguientes funciones (Tabla \ref{tab:read_data}).}

\begin{table}[h]
\centering
\begin{tabular}{c|p{6cm}|p{5cm}}
\hline
\textbf{Función} & \textbf{Uso} & \textbf{Ejemplo} \\
\hline

\multirow{2}{*}{$\code{read\_csv}$}
& \multirow{2}{6cm}{Cargar datos delimitados por comas desde un archivo local o una URL}
& \multirow{2}{*}{$\code{pd.read\_csv("datos.csv")}$}  \\
&  \\
\hline

\multirow{2}{*}{$\code{read\_excel}$}
& \multirow{2}{6cm}{Cargar datos tabulares desde un archivo Excel}
& \multirow{2}{*}{$\code{pd.read\_excel("datos.xlsx")}$} \\
&  \\
\hline

\multirow{2}{*}{$\code{read\_pickle}$}
& \multirow{2}{6cm}{Cargar datos serializados en formato pickle}
& \multirow{2}{*}{$\code{pd.read\_pickle("datos.pkl")}$} \\
&  \\
\hline

\multirow{2}{*}{$\code{read\_json}$}
& \multirow{2}{6cm}{Cargar datos desde archivos JSON}
& \multirow{2}{*}{$\code{pd.read\_json("datos.json")}$} \\
&  \\
\hline

\end{tabular}
\caption{Funciones de Pandas para carga de datos}
\label{tab:read_data}
\end{table}



\end{frame}


\begin{frame}{Como guardar datos con Pandas ?}


Una vez que hemos modificado datos en un objeto tipo DataFrame o Serie, podemos guardar este usando las siguientes funciones (Tabla \ref{tab:save_data}).

\begin{table}[h]
\centering
\begin{tabular}{c|p{6cm}|p{5cm}}
\hline
\textbf{Función} & \textbf{Uso} & \textbf{Ejemplo} \\
\hline

\multirow{2}{*}{$\code{to\_csv}$}
& \multirow{2}{6cm}{Guardar un DataFrame en un archivo delimitado por comas (CSV)}
& \multirow{2}{*}{$\code{df.to\_csv("datos.csv")}$} \\
&  \\
\hline

\multirow{2}{*}{$\code{to\_excel}$}
& \multirow{2}{6cm}{Guardar un DataFrame en un archivo Excel}
& \multirow{2}{*}{$\code{df.to\_excel("datos.xlsx")}$} \\
&  \\
\hline

\multirow{2}{*}{$\code{to\_pickle}$}
& \multirow{2}{6cm}{Serializar y guardar un DataFrame en formato pickle}
& \multirow{2}{*}{$\code{df.to\_pickle("datos.pkl")}$} \\
&  \\
\hline

\multirow{2}{*}{$\code{to\_json}$}
& \multirow{2}{6cm}{Guardar un DataFrame en un archivo JSON}
& \multirow{2}{*}{$\code{df.to\_json("datos.json")}$} \\
&  \\
\hline

\end{tabular}
\caption{Funciones de Pandas para guardar datos}
\label{tab:save_data}
\end{table}



\end{frame}

\begin{frame}{Vamos a ver un ejemplo practico en Python Collab}
    Vamos a explorar un dataset de Pokemons: \url{https://colab.research.google.com/drive/1X7xoTisfFrGZFfrQ9ROckTaYzv8Xw8TB?usp=sharing}.

\begin{figure}
    \centering
    \includegraphics[width=0.5\linewidth]{days/03/image_5/pokemon.png}
    \caption{Pokemones de primera generación.}
    \label{fig:placeholder}
\end{figure}
    
\end{frame}





\begin{frame}{Referencias:}
    \begin{itemize}
        \item Wes McKinney. (2022). Python for Data Analysis. Third Edition.
    \end{itemize}
\end{frame}


\end{document}