\documentclass[aspectratio=169]{beamer}
\usepackage{borelian}
\usepackage{multirow}

\begin{document}
\classtitle{6}{Visualización de información}{7 de enero de 2026}

\begin{frame}{Visualización de información}

\begin{itemize}
    \item La información puede almacenarse en diferentes formatos: \textbf{grafos, arboles, tablas, diccionarios, entre otros.}
    \item En este curso nos enfocaremos en datos tabulares (tablas). Este es el formato mas simple y utilizado.
    \item En la clase anterior vimos como abrir un datasets de este tipo usando Pandas. Ahora, la idea es visualizar estos datos. 
\end{itemize}

Pero antes de eso, por que nos gustaria visualizar los datos ? 

\begin{itemize}
    \item Como vimos en la clase de estadística, los datasets pueden ser bastante complejos para analizar dato a dato. 
    \item Las tecnicas de visualización nos permiten encontrar patrones o tendencias en los datos.
    \item Se puede decir que la visualización de datos nos permite contar una historia de estos.
    \item Esto nos ayuda a tomar decisiones sobre los datos.
\end{itemize}

\vspace{5mm}

\begin{itemize}
    \item \mintinline{python}{Matplotlib}
    \item \mintinline{python}{Seaplot}
\end{itemize}
\end{frame}

\begin{frame}[fragile]{Matplotlib}

Matplotlib es la librería mas popular para visualizar datos en Python. Esta libreria tiene funciones acopladas al paquete de Pandas.

\begin{minted}{python}
import matplotlib.pyplot as plt
\end{minted}

\begin{table}[]
    \centering
    \begin{tabular}{c|c|c}
    Función & Uso & Ejemplo  \\
    \hline
    \mintinline{python}{plot} & Gráfico de linea & \mintinline{python}{ plt.plot(valores) } \\
    \hline
    \mintinline{python}{scatter} & Gráfico de puntos & \mintinline{python}{ plt.scatter(valores) } \\
    \hline
    \mintinline{python}{bar} & Gráfico de barras & \mintinline{python}{ plt.bar(etiquetas, conteos) } \\
    \hline
    \mintinline{python}{boxplot} & Gráfico de caja & \mintinline{python}{ plt.boxplot(valores) } \\
    \hline
    \mintinline{python}{histogram} & Histograma & \mintinline{python}{ plt.hist(valores) } \\
    \hline
    \mintinline{python}{imshow} & Histograma & \mintinline{python}{ plt.plot(valores) } \\
    
    \end{tabular}
    \caption{Funciones en matplotlib.}
    %\label{tab:placeholder}
\end{table}

Analizaremos cada una de estas funciones y veremos como aplicarlas en un dataset.

\tiny \url{https://colab.research.google.com/drive/1kKzcxZf8Ova4Rd2Yrb-Zui5163hWYb6G?usp=sharing}

\end{frame}


%%%%%%%%%%%%%%%%%%%%%%%%%%%%%%%%%%%%%%%%%%%%%%%%%%%%%%%%%%%%

\begin{frame}[fragile]{Matplotlib: plot}

El gráfico de líneas se usa para visualizar la evolución de una variable continua,
generalmente respecto al tiempo o al orden de observación.

\begin{itemize}
    \item Se usan para analizar tendencias temporales y comparación la evoluación de variables.
\end{itemize}


Los ejes x e y representan los valores que adoptan las variables analizadas.

\vspace{3mm}

\begin{minted}{python}
>>> import matplotlib.pyplot as plt

>>> y = [1, 3, 2, 5, 4]
>>> plt.plot(y)
>>> plt.show()
\end{minted}

\end{frame}

%%%%%%%%%%%%%%%%%%%%%%%%%%%%%%%%%%%%%%%%%%%%%%%%%%%%%%%%%%%%


\begin{frame}[fragile]{Matplotlib: scatter}

El gráfico de dispersión muestra la relación entre dos variables numéricas.

\begin{itemize}
    \item Permite detectar correlaciones, identificar patrones, clusteres y encontrar outliers.
\end{itemize}

Los ejes x e y representan los valores que adoptan las variables analizadas.

\vspace{3mm}

\begin{minted}{python}
x = [1, 2, 3, 4, 5]
y = [2, 1, 3, 5, 4]

# plot de (1,2), (2,1), ... (5,4)
plt.scatter(x, y) 
plt.show()

\end{minted}

\end{frame}



%%%%%%%%%%%%%%%%%%%%%%%%%%%%%%%%%%%%%%%%%%%%%%%%%%%%%%%%%%%%


\begin{frame}[fragile]{Matplotlib: hist }

El histograma muestra la distribución de una variable numérica.

\begin{itemize}
    \item Se utiliza para analizar la forma de la distribución, asimetria y concentración de valores.
\end{itemize}

En general, el eje x representa la variable analizada y el eje y el numero de veces que ocurre esa variable (frecuencia).

\vspace{3mm}

\begin{minted}{python}
import numpy as np
# 1000 datos random
temperatura_santiago = np.random.randn(1000) + 25 #media 25 y desviacion estandar 1
plt.hist(temperatura_santiago, bins=20)
plt.xlabel("Temperatura [C]")
plt.ylabel("Frecuencia [-]")
plt.show()
\end{minted}

\end{frame}


%%%%%%%%%%%%%%%%%%%%%%%%%%%%%%%%%%%%%%%%%%%%%%%%%%%%%%%%%%%%

\begin{frame}[fragile]{Matplotlib: boxplot}

El boxplot resume una distribución mediante cuartiles y valores extremos.

\begin{itemize}
    \item Se usa para comparar las distribuciones de diversas categorías en función de sus cuartiles y outliers.
\end{itemize}

Requiere un atributo categorico (eje x) y un atributo cuantitativo (eje y). 

\vspace{3mm}

\begin{minted}{python}
datos = [1, 2, 2, 3, 4, 10]
plt.boxplot(datos)
plt.show()
\end{minted}

\end{frame}

%%%%%%%%%%%%%%%%%%%%%%%%%%%%%%%%%%%%%%%%%%%%%%%%%%%%%%%%%%%%


\begin{frame}[fragile]{Matplotlib: imshow}

\mintinline{python}{imshow} se utiliza para visualizar datos matriciales como imágenes.

\begin{itemize}
    \item Imágenes.
    \item Matrices.
    \item Mapas.
\end{itemize}

\vspace{3mm}

\begin{minted}{python}
import numpy as np

# matrix de 10 filas x 10 columnas
matriz = np.random.rand(10, 10)
plt.imshow(matriz)
plt.colorbar()
plt.show()
\end{minted}

\end{frame}



\begin{frame}[fragile]{Aplicación sobre Iris Dataset}

\begin{itemize}
   \item Dataset con 150 datos tabulares.
   \item Cada instancia (fila) representa una flor.
   \item En este dataset tiene 5 columnas: \textbf{sepallength, sepalwidth, petallength, petalwidth, class}.
   \item Las clases (class) son categoricas: \textbf{setosa, versicolor, virginica}.
\end{itemize}

\begin{figure}[H]
\centering
\includegraphics[width=0.2\linewidth]{days/03/image/iris.png}
\caption{Ejemplo de Series.}
\end{figure}

%Se les ocurre que tipo de modelo predictivo podriamos generar con estos datos ?

Por ahora nos limitaremos a visualizar los datos.

%Eso lo veremos en las siguientes clases, por ahora nos limitaremos a visualizar los datos.

\tiny \url{https://colab.research.google.com/drive/1kKzcxZf8Ova4Rd2Yrb-Zui5163hWYb6G?usp=sharing}

\end{frame}

%%%%%%%%%%%%%%%%%%%%%%%%%%%%%%%%%%%%%%%%%%%%%%%%%%%%%%%%%%%%


\begin{frame}[fragile]{Seaborn}

Seaborn es una librería de visualización que permite generar gráficos de manera simple, rápida y con muchos estilos. Con Matplotlib podemos lograr lo mismo pero requiere mayor trabajo.

\begin{minted}{python}
import seaborn as sns
\end{minted}

\begin{itemize}
    \item \textbf{lineplot}:  
    Gráfico de líneas con estimación estadística e intervalos de confianza. \\  
    \mintinline{python}{sns.lineplot(data=df, x="x", y="y")}

    \item \textbf{scatterplot}:  
    Gráfico de dispersión con agrupación por color, tamaño o estilo.  \\
    \mintinline{python}{sns.scatterplot(data=df, x="x", y="y", hue="clase")}

    \item \textbf{histplot}:  
    Histograma y densidad de una variable. \\
    \mintinline{python}{sns.histplot(data=df, x="x", kde=True)}

    \item \textbf{boxplot}:  
    Comparación de distribuciones mediante cajas. \\ 
    \mintinline{python}{sns.boxplot(data=df, x="grupo", y="valor")}

    \item \textbf{pairplot}:  
    Relación entre todas las variables numéricas.\\  
    \mintinline{python}{sns.pairplot(df, hue="clase")}

\end{itemize}

\textbf{Ahora continuaremos viendo como utilizar estas funciones en el dataset Iris.}

\end{frame}

\begin{frame}{Referencias:}
    \begin{itemize}
        \item Wes McKinney. (2022). Python for Data Analysis. Third Edition.
        \item \url{https://seaborn.pydata.org/tutorial/function_overview.html}
        \item 
    \end{itemize}
\end{frame}


\end{document}