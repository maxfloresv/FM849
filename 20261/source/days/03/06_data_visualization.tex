\documentclass[aspectratio=169,handout]{beamer}
\usepackage{borelian}
\usepackage{multirow}

\begin{document}
\classtitle{6}{Visualización de información}{7 de enero de 2026}

\begin{frame}{Visualización de información}

\begin{itemize}
    \item La información puede almacenarse en diferentes formatos: \textbf{grafos, arboles, tablas, diccionarios, entre otros.}
    \pause
    \item En este curso nos enfocaremos en datos tabulares (tablas). Este es el formato mas simple y utilizado.
    \pause
    \item En la clase anterior vimos como abrir un dataset tabular usando Pandas. Ahora, la idea es visualizar estos datos.
    \pause 
\end{itemize}

Pero antes de eso, por que nos gustaria visualizar los datos ? 

\begin{itemize}
    \item Como vimos en la clase de estadística, los datasets pueden ser bastante complejos para analizar fila por fila. 
    \pause
    \item Las tecnicas de visualización nos permiten encontrar patrones o tendencias en los datos.
    \pause
    \item La visualización con gráficos nos permite responder a preguntas asociadas a los datos.
\end{itemize}

\vspace{5mm}

\end{frame}


%%%%%%%%%%%%%%%%%%%%%%%%%%%%%%%%%%%%%%%%%%%%%%%%%%%%%%%%%%%%%%%%%%%%%%%%%%%%%%%%%%%%%%%%%%%%%%%%

\begin{frame}[fragile]{Matplotlib vs Seaborn}
    En esta clase trabajaremos con dos paquetes para visualizar datos.

    \begin{itemize}
        \item \textbf{Matplotlib}: Librería de uso general para construir todo tipo de gráficos.
        \begin{itemize}
            \item Es muy flexible, pero requiere más código.
            \item Recordar importar: \mintinline{python}{import matplotlib.pyplot as plt}
            \item Las funciones de matplotlib reciben los vectores de datos a plotear como argumentos (no un DataFrame).
        \end{itemize}

        \item \textbf{Seaborn}: Librería orientada al análisis de datos.
        \begin{itemize}
            \item Está pensada para trabajar directamente con \textbf{DataFrames}.
            \item Recordar importar: \mintinline{python}{import seaborn as sns}
        \end{itemize}
    \end{itemize}

\end{frame}


\begin{frame}[fragile]{Matplotlib vs Seaborn}

    En Matplotlib, los gráficos se construyen indicando:
    \begin{itemize}
        \item Directamente los \textbf{datos a graficar} (arreglos, listas o Series).
        \item No existe un argumento \mintinline{python}{data} que agrupe los datos.
        \item Los argumentos (\mintinline{python}{x}, \mintinline{python}{y}, etc.) corresponden a los \textbf{arreglos de valores} y no a los nombres de las columnas.
    \end{itemize}

    En Seaborn, los gráficos se construyen indicando:
    \begin{itemize}
        \item \mintinline{python}{data}: el \textbf{DataFrame} que contiene los datos.
        \item El resto de los argumentos (\mintinline{python}{x}, \mintinline{python}{y}, \mintinline{python}{hue}, etc.) son los \textbf{nombres de las columnas} de ese DataFrame que se quieren graficar.
    \end{itemize}

    Esto permite crear gráficos de forma simple, clara y eficiente, sin necesidad de extraer manualmente los datos como ocurre habitualmente en Matplotlib.
\end{frame}


%%%%%%%%%%%%%%%%%%%%%%%%%%%%%%%%%%%%%%%%%%%%%%%%%%%%%%%%%%%%%%%%%%%%%%%%%%%%%%%%%%%%%%%%%%%%%%%%


%\tiny \url{https://colab.research.google.com/drive/1kKzcxZf8Ova4Rd2Yrb-Zui5163hWYb6G?usp=sharing}
%%%%%%%%%%%%%%%%%%%%%%%%%%%%%%%%%%%%%%%%%%%%%%%%%%%%%%%%%%%%

\begin{frame}[fragile]{Gráficos de Línea}
\begin{itemize}
    \item Los ejes x e y representan los valores que adoptan las variables analizadas.
    \item Se usan para analizar tendencias temporales y comparación la evolución de variables.
    \item Las marcas son puntos que se conectan por líneas.
    \item La posición vertical expresa un valor cuantitativo, mientras la posición horizontal contiene las llaves ordenadas.
\end{itemize}

%Se puede usar para encontrar tendencias y patrones. Las llaves deben estar ordenadas para que el gráfico tenga sentido.

\begin{figure}[H]
\centering
\includegraphics[width=0.5\linewidth]{days/03/images/line_example.png}
\caption{Gráfico de línea mostrando la evolución de la temperatura un año.}
\end{figure}
\end{frame}


\begin{frame}[fragile]{Ejemplo de gráfico de línea (Matplotlib)}

 \begin{columns}
    \column{0.5\textwidth}
    Con \texttt{plt.plot}:
    \begin{minted}{python}
    import matplotlib.pyplot as plt
    >>> x = [0, 1, 2, 3, 4]
    >>> y = [1, 3, 2, 5, 4]
    >>> # plot (0,1), (1,3), ... (4,4)
    >>> plt.plot(x,y)
    >>> plt.ylabel("y")
    >>> plt.show()
    \end{minted}

\column{0.5\textwidth}

    \begin{figure}[H]
    \centering
    \includegraphics[width=0.8\linewidth]{days/03/images/plt_line.png}
    \caption{Gráfico de línea en matplotlib. Muestra los valores de la lista x (eje x) y lista y (eje y).}
    \end{figure}
\end{columns}

Mas ejemplos: \url{https://matplotlib.org/stable/api/_as_gen/matplotlib.pyplot.plot.html}

\end{frame}


\begin{frame}[fragile]{Ejemplo de gráfico de línea (Seaborn)}
\small
\begin{columns}
    \column{0.4\textwidth}
    Con \texttt{sns.lineplot}:
    \begin{minted}{python}
    >>> import seaborn as sns
    >>> # DataFrame de vuelos
    >>> df = sns.load_dataset("flights")
    >>> sns.lineplot(data=df,
                     x="year",
                     y="passengers")
    >>> plt.show()
    \end{minted}

    \column{0.6\textwidth}
    
    \begin{figure}[H]
        \centering
        \includegraphics[width=0.75\linewidth]{days/03/images/lineplot_sea.png}
        \caption{Gráfico de línea que muestra el numero de vuelos (eje y) según el año (eje x). El área sombreada representa un intervalo de confianza.}
    \end{figure}
\end{columns}

\vspace{-3mm}
Mas ejemplos: \url{https://seaborn.pydata.org/generated/seaborn.lineplot.html}

\end{frame}



%%%%%%%%%%%%%%%%%%% GRAFICO DE DISPERSION %%%%%%%%%%%%%%%%%%%%%%%%

\begin{frame}{Gráfico de Dispersión}
\small
\begin{columns}
    \column{0.5\textwidth}
    \begin{itemize}
        \item El gráfico de dispersión muestra la relación entre dos variables numéricas.
        \item Permite detectar correlaciones, identificar patrones, clusteres y encontrar outliers.
        %\item La marca son los puntos.
        \item Los canales son las posiciones: horizontal y vertical. Los ejes x e y representan los valores que adoptan las variables analizadas.
    \end{itemize}

    \column{0.5\textwidth}
    
    \begin{figure}[H]
\centering
\includegraphics[width=0.8\linewidth]{days/03/images/scatter_mpg.png}
\caption{Gráfico de dispersión entre millas por galon (MPG, eje y) y caballos de fuerza (horsepower, eje x) en autos.}
\end{figure}

\end{columns}

\end{frame}


\begin{frame}[fragile]{Ejemplo de gráfico de dispersión (Matplotlib)}
\begin{columns}
    \column{0.5\textwidth}
    Con \texttt{plt.scatter}:
\begin{minted}{python}
>>> import matplotlib.pyplot as plt
>>> x = [1, 2, 3, 4, 5]
>>> y = [2, 1, 3, 5, 4]
>>> # plot de (1,2), (2,1), ... , (5,4)
>>> plt.scatter(x, y)
>>> plt.ylabel("y")
>>> plt.xlabel("x")
>>> plt.show()
\end{minted}
    \column{0.5\textwidth}
\begin{figure}[H]
\centering
\includegraphics[width=0.8\linewidth]{days/03/images/plt_scatter.png}
\caption{Gráfico de dispersión en Matplotlib. Muestra los valores de la lista x (eje x) y lista y (eje y).}
\end{figure}
\end{columns}

Mas ejemplos: \url{https://matplotlib.org/stable/api/_as_gen/matplotlib.pyplot.scatter.html}

\end{frame}


\begin{frame}[fragile]{Ejemplo de gráfico de dispersión (Seaborn)}
\small

\begin{columns}
    \column{0.4\textwidth}
    Con \texttt{sns.scatterplot}:
    \begin{minted}{python}
>>> import seaborn as sns
>>> #DataFrame de propinas
>>> df = sns.load_dataset("tips")
>>> sns.scatterplot(data=df,
                    x="total_bill",
                    y="tip")
>>> plt.show()
\end{minted}

\column{0.6\textwidth}
\begin{figure}[H]
\centering
\includegraphics[width=0.7\linewidth]{days/03/images/scatter_plot.png}
\caption{Gráfico de dispersión entre pago total (total bill, eje x) y propina (tip, eje y).}
\end{figure}

\end{columns}

Mas ejemplos: \url{https://seaborn.pydata.org/generated/seaborn.scatterplot.html}

\end{frame}



%%%%%%%%%%%%%%%%%%% GRAFICO DE HISTOGRAMA %%%%%%%%%%%%%%%%%%%%%%%%


\begin{frame}[fragile]{Grafico de Barras.}

\begin{itemize}
    \item Requiere 1 atributo categórico y 1 cuantitativo.
    \item Los canales que codifican información incluyen: el largo de la barra para expresar un valor cuantitativo y una separación en el espacio para representar otra categoría.
    \item Se puede usar para comparar categorías y encontrar casos extremos.
\end{itemize}

\begin{figure}[H]
\centering
\includegraphics[width=0.4\linewidth]{days/03/images/barplot.png}
\caption{Edad promedio según la clase en el barco Titanic.}
\end{figure}

\end{frame}



\begin{frame}[fragile]{Ejemplo de gráfico de barras (Matplotlib)}
\begin{columns}
    \column{0.5\textwidth}
    Con \texttt{plt.bar}:
\begin{minted}{python}
>>> import matplotlib.pyplot as plt
>>> categorias = ["A", "B", "C", "D"]
>>> valores = [23, 45, 12, 30]
>>> plt.bar(categorias, valores)
>>> plt.ylabel("Valor")
>>> plt.xlabel("Categoría")
>>> plt.show()
\end{minted}
    \column{0.5\textwidth}
\begin{figure}[H]
\centering
\includegraphics[width=0.7\linewidth]{days/03/images/plt_bar.png}
\caption{Gráfico de barras generado en Matplotlib. Muestra valores (eje y) para cada categoria (eje x).}
\end{figure}
\end{columns}

Mas ejemplos: \url{https://matplotlib.org/stable/api/_as_gen/matplotlib.pyplot.bar.html}

\end{frame}


\begin{frame}[fragile]{Ejemplo de gráfico de barras (Seaborn)}
\small

\begin{columns}
    \column{0.4\textwidth}
    Con \texttt{sns.barplot}:
    \begin{minted}{python}
>>> import seaborn as sns
>>> #DataFrame de pinguinos
>>> df = sns.load_dataset("penguins")
>>> sns.barplot(df,
                x="island",
                y="body_mass_g")
>>> plt.show()
\end{minted}

\column{0.6\textwidth}
\begin{figure}[H]
    \centering
    \includegraphics[width=0.77\linewidth]{days/03/images/barplot_seaborn.png}
    \caption{Gráfico de barras mostrando el peso promedio de pinguinos (variable numerica en eje y) en islas (variable categorica).}
\end{figure}

\end{columns}

Mas ejemplos: \url{https://seaborn.pydata.org/generated/seaborn.barplot.html}

\end{frame}


%%%%%%%%%%%%%%%%%%% GRAFICO DE HISTOGRAMA %%%%%%%%%%%%%%%%%%%%%%%%

%\begin{itemize}
%     \item Un histograma es un gráfico de barras que representa una variable cuantitativa, agrupando los datos en intervalos (rangos).
%    \item Cada barra corresponde a un intervalo de valores, cuyos límites pueden mostrarse explícitamente o inferirse a partir de la escala del eje horizontal.
%    \item Se utiliza para analizar la forma de la distribución, su asimetría y la concentración de valores.
%    \item El eje $y$ representa el número de observaciones que caen en cada intervalo (frecuencia o conteo).
%    %\item En general, el eje x representa la variable analizada y el eje y el numero de veces que ocurre esa variable (frecuencia).
%\end{itemize}

\begin{frame}[fragile]{Histograma }
\small

\begin{columns}
    \column{0.5\textwidth}
    \begin{itemize}
        \item Un histograma representa una variable cuantitativa agrupando los datos en intervalos.
        \item Cada barra corresponde a un rango de valores del eje horizontal.
        \item Permite analizar la forma de la distribución, su asimetría y concentración.
        \item El eje $y$ muestra la frecuencia (número de observaciones) en cada intervalo.
    \end{itemize}

    \column{0.5\textwidth}
    \begin{figure}[H]
\centering
\includegraphics[width=0.6\linewidth]{days/03/images/hist.png}
\caption{Histograma que muestra las distribucion del largo del sepalo en dataset Iris.}
\end{figure}

\end{columns}

\end{frame}



\begin{frame}[fragile]{Ejemplo de histograma (Matplotlib)}
\begin{columns}
\column{0.5\textwidth}
Con \texttt{plt.hist}:
\begin{minted}{python}
>>> import matplotlib.pyplot as plt
>>> import numpy as np
>>> temperatura = [2,33,24,25,12,20,0,23] 
>>> plt.hist(temperatura)
>>> plt.xlabel("Temperatura [°C]")
>>> plt.ylabel("Conteo [-]")
>>> plt.show()
\end{minted}
\column{0.5\textwidth}
\begin{figure}[H]
\centering
\includegraphics[width=0.8\linewidth]{days/03/images/histograma_plt.png}
\caption{Histograma en matplotlib. Muestra la frecuencia (eje y) de veces que ocurre valores de temperatura (eje x). }
\end{figure}
\end{columns}

Mas ejemplos: \url{https://matplotlib.org/stable/api/_as_gen/matplotlib.pyplot.hist.html}

\end{frame}



\begin{frame}[fragile]{Ejemplo de histograma (Seaborn)}
\small

\begin{columns}
    \column{0.35\textwidth}
    Con \texttt{sns.histplot}:
    \begin{minted}{python}
>>> import seaborn as sns
>>> # DataFrame de pinguinos
>>> df = sns.load_dataset("penguins") 
>>> sns.histplot(data=df,
                 x="flipper_length_mm")
>>> plt.show()
\end{minted}
%>>> #generacion de histograma sobre el largo de aleta de pinguinos
%>>> #carga de dataframe de pinguinos

\column{0.5\textwidth}
\begin{figure}[H]
    \centering
    \includegraphics[width=0.75\linewidth]{days/03/images/hist_flipper.png}
    \caption{Histograma del largo de la aleta de pingüinos (\texttt{flipper\_length\_mm}), que muestra la frecuencia (eje y) de los distintos valores de largo de aleta observados (eje x).}
\end{figure}

\end{columns}

    Mas ejemplos: \url{https://seaborn.pydata.org/generated/seaborn.histplot.html}
\end{frame}




%%%%%%%%%%%%%%%%%%% GRAFICO DE BOXPLOT %%%%%%%%%%%%%%%%%%%%%%%%


\begin{frame}[fragile]{Matplotlib: boxplot}

\begin{columns}
\column{0.5\textwidth}
\begin{itemize}
        \item El boxplot resume una distribución considerando 5 valores obtenidos de la serie de datos: minimo, cuartil 1 (Q1), cuartil 2 (mediana), cuartil 3 (Q3) y maximo. 
        \item Se usa para comparar las distribuciones entre columnas de un DataFrame en función de sus cuartiles y outliers.
        \item Requiere un atributo categorico (eje x) y un atributo cuantitativo (eje y). 
        \item El principal canal de codificación es el largo de la caja (rango intercuartil) y las lineas horizontales (asociadas a outliers).
\end{itemize}

\column{0.5\textwidth}
\begin{figure}[H]
\centering
\includegraphics[width=0.55\linewidth]{days/03/images/boxplot_draw.png}
\caption{Componentes de boxplot. IQR es el rango intercuartil.}
\end{figure}
\end{columns}


\end{frame}

\begin{frame}[fragile]{Matplotlib: boxplot}

\begin{columns}
\column{0.5\textwidth}
\begin{itemize}
        \item El boxplot resume una distribución considerando 5 valores obtenidos de la serie de datos: minimo, cuartil 1 (Q1), cuartil 2 (mediana), cuartil 3 (Q3) y maximo. 
        \item Se usa para comparar las distribuciones entre columnas de un DataFrame en función de sus cuartiles y outliers.
        \item Requiere un atributo categorico (eje x) y un atributo cuantitativo (eje y). 
        \item El principal canal de codificación es el largo de la caja (rango intercuartil) y las lineas horizontales (asociadas a outliers).
\end{itemize}

\column{0.5\textwidth}
\begin{figure}[H]
\centering
\includegraphics[width=0.65\linewidth]{days/03/images/boxplot2.png}
\caption{Ejemplo de boxplot para comparar categorias (A, B y C).}
\end{figure}
\end{columns}


\end{frame}


\begin{frame}[fragile]{Ejemplo de boxplot en Matplotlib}
\begin{columns}
\column{0.5\textwidth}
Con \texttt{plt.boxplot}:
\begin{minted}{python}
>>> import matplotlib.pyplot as plt
>>> temperatura = [2,33,24,25,12,20,0,23,50] 
>>> plt.boxplot(temperatura)
>>> plt.ylabel("Temperatura")
>>> plt.show()
\end{minted}
\column{0.5\textwidth}
\begin{figure}[H]
\centering
\includegraphics[width=0.7\linewidth]{days/03/images/boxplot_plt.png}
\caption{Boxplot de la variable temperatura (eje y), que muestra el valor mínimo, el máximo y los cuartiles primero (Q1), segundo (mediana) y tercero (Q3).}
\end{figure}
\end{columns}

Mas ejemplos: \url{https://matplotlib.org/stable/api/_as_gen/matplotlib.pyplot.boxplot.html}

\end{frame}


\begin{frame}[fragile]{Ejemplo de boxplot en Seaborn}
\begin{columns}
\column{0.5\textwidth}
Con \texttt{sns.boxplot}:
\begin{minted}{python}
>>> import matplotlib.pyplot as plt
>>> import seaborn as sns
>>> df = sns.load_dataset('titanic')
>>> sns.boxplot(data=df,x="class",y="age")
>>> plt.ylabel("Edad")
>>> plt.xlabel("Clase")
>>> plt.show()
\end{minted}
\column{0.6\textwidth}
\begin{figure}[H]
\centering
\includegraphics[width=0.6\linewidth]{days/03/images/box_seaborn.png}
\caption{Boxplot de distribuciones de edades según clase en embarcación Titanic. Muestra los valores de edades mínimo, el máximo y los cuartiles primero (Q1), segundo (mediana) y tercero (Q3) separado por clase.}
\end{figure}
\end{columns}


Mas ejemplos: \url{https://seaborn.pydata.org/generated/seaborn.boxplot.html}


\end{frame}





%%%%%%%%%%%%%%%%%%% GRAFICO DE HEATMAP %%%%%%%%%%%%%%%%%%%%%%%%


\begin{frame}[fragile]{Heatmaps}
\small

\begin{columns}
    \column{0.5\textwidth}
    \begin{itemize}
        \item Un heatmap es una visualización que representa valores numéricos mediante una escala de colores.
        \item Se utiliza para analizar patrones, concentraciones y relaciones en datos bidimensionales.
        \item Permite identificar rápidamente zonas de valores altos y bajos.
        \item Es comúnmente usado para visualizar matrices de correlación, distancias o frecuencias.
        \item El color es el principal canal de codificación del valor.
    \end{itemize}

    \column{0.5\textwidth}
    \begin{figure}[H]
        \centering
        \includegraphics[width=0.65\linewidth]{days/03/images/heatmap.png}
        \caption{Ejemplo de heatmap. El eje x representa diferentes granjeros; el eje y representa verduras. El color en la matriz representa la cantidad de verduras cosechadas (toneladas/año) para combinacion de verdura y granjero.}
    \end{figure}
\end{columns}

\end{frame}

\begin{frame}[fragile]{Heatmap con Matplotlib (imshow)}
\small
\begin{columns}
\column{0.5\textwidth}
Con \texttt{plt.imshow}:
\begin{minted}{python}
>>> import matplotlib.pyplot as plt
>>> import numpy as np
>>> data = np.random.rand(4, 4)
>>> plt.imshow(data, cmap="viridis")
>>> plt.colorbar(label="Valor")
>>> plt.xlabel("Indice columna")
>>> plt.ylabel("Indice fila")
>>> plt.title("Heatmap usando imshow")
>>> plt.show()
\end{minted}

\column{0.5\textwidth}
\begin{figure}[H]
\centering
\includegraphics[width=0.75\linewidth]{days/03/images/heatmap_plt.png}
\caption{Heatmap generado con \texttt{imshow} en Matplotlib. Los colores representan los valores numéricos de una matriz de 4×4 generada aleatoriamente, donde cada celda se codifica mediante una escala de color.}
\end{figure}
\end{columns}

Mas ejemplos: \url{https://matplotlib.org/stable/api/_as_gen/matplotlib.pyplot.imshow.html}

\end{frame}


\begin{frame}[fragile]{Heatmap con Seaborn}
\small

\begin{columns}
    
    \column{0.45\textwidth}
    Con \texttt{sns.heatmap}:
\begin{minted}{python}
>>> import seaborn as sns
>>> import matplotlib.pyplot as plt
>>> df = sns.load_dataset("iris")
>>> corr = df.corr(numeric_only=True)
>>> sns.heatmap(corr, annot=True,
                cmap="coolwarm")
>>> plt.title("Matriz de correlación")
>>> plt.show()
\end{minted}
Mas ejemplos: \url{https://seaborn.pydata.org/generated/seaborn.heatmap.html}

\column{0.55\textwidth}
\begin{figure}[H]
    \centering
    \includegraphics[width=0.8\linewidth]{days/03/images/correlacion_seaborn.png}
    \caption{Heatmap de la matriz de correlación del dataset Iris, donde los colores indican la intensidad y el signo de la correlación lineal entre las variables numéricas, y los valores anotados corresponden a los coeficientes de correlación.}
\end{figure}

\end{columns}



\end{frame}



\begin{frame}{Vamos a ver un ejemplo práctico en Python Collab}
    Exploremos un dataset: \url{https://colab.research.google.com/drive/1Tdz4RbEasHZ69BbEkGKErjuKW-TIuN90?usp=sharing}.

\begin{figure}
    \centering
    \includegraphics[width=0.5\linewidth]{days/03/images/pokemon.png}
    \caption{Pokemones de primera generación.}
    \label{fig:placeholder}
\end{figure}
    
\end{frame}




%%%%%%%%%%%%%%%%%%% GRAFICO DE HEATMAP %%%%%%%%%%%%%%%%%%%%%%%%
\begin{frame}[fragile]{Matplotlib (Resumen)}
%Matplotlib es la librería mas popular para visualizar datos en Python. % Esta libreria tiene funciones acopladas al paquete de Pandas.

\begin{minted}{python}
import matplotlib.pyplot as plt # podemos importarla de esta forma
\end{minted}

\begin{table}[]
    \centering
    \begin{tabular}{c|c|c}
    Función & Uso & Ejemplo  \\
    \hline
    \mintinline{python}{plot} & Gráfico de línea & \mintinline{python}{ plt.plot(valores) } \\
    \hline
    \mintinline{python}{scatter} & Gráfico de puntos & \mintinline{python}{ plt.scatter(valores_x,valores_y) } \\
    \hline
    \mintinline{python}{bar} & Gráfico de barras & \mintinline{python}{ plt.bar(etiquetas, valores) } \\
    \hline
    \mintinline{python}{boxplot} & Gráfico de caja & \mintinline{python}{ plt.boxplot(valores) } \\
    \hline
    \mintinline{python}{histogram} & Histograma & \mintinline{python}{ plt.hist(valores) } \\
    \hline
    \mintinline{python}{imshow} & Gráfico heatmap & \mintinline{python}{ plt.imshow(matriz) } \\
    
    \end{tabular}
    \caption{Funciones en matplotlib.}
    %\label{tab:placeholder}
\end{table}

\end{frame}

\begin{frame}[fragile]{Seaborn (Resumen)}
\small

\begin{minted}{python}
import seaborn as sns # podemos importarla de esta forma
\end{minted}

\begin{itemize}
    \item \textbf{lineplot}:  
    Gráfico de que muestra la relación entre la columna \textbf{'x'} e \textbf{'y'} de un dataframe. \\  
    \mintinline{python}{sns.lineplot(data=dataframe, x="x", y="y")}

    \item \textbf{scatterplot}:  
    Gráfico de dispersión entre columna \textbf{'x'} e \textbf{'y'} de un dataframe.  \\
    \mintinline{python}{sns.scatterplot(data=dataframe, x="x", y="y")}

    \item \textbf{histplot}:  
    Histograma de la columna \textbf{'x'} de un dataframe. \\
    \mintinline{python}{sns.histplot(data=dataframe, x="x")}

    \item \textbf{boxplot}:  
    Comparación de distribuciones de la columna \textbf{'valor'} en un valor. \\ 
    \mintinline{python}{sns.boxplot(data=dataframe, x="grupo", y="valor")}

    \item \textbf{heatmap}:  
    Visualización matricial donde los colores representan los valores numéricos de una tabla, comúnmente usada para analizar relaciones entre variables (por ejemplo, matrices de correlación). \\
    \mintinline{python}{sns.heatmap(dataframe, annot=True)}

    %\item \textbf{pairplot}:  
    %Relación entre todas las variables numéricas en un dataframe coloreadas segun la columna \textbf{'clase'}.\\  
    %\mintinline{python}{sns.pairplot(dataframe, hue="clase")}

\end{itemize}

\end{frame}




\begin{frame}{Referencias:}
    \begin{itemize}
        \item Wes McKinney. (2022). Python for Data Analysis. Third Edition.
        \item \url{https://seaborn.pydata.org/tutorial/function_overview.html} 
    \end{itemize}
\end{frame}


\end{document}






%%%%%%%%%%%%%%%%%%%%%%%%%%%%%%%%%%%%%%%%%%%%%%%%%%%%%%%%%%%%%%%%%%%%%%%%%%

















%%%%%%%%%%%%%%%%%%%%%%%%%%%%%%%%%%%%%%%%%%%%%%%%%%%%%%%%%%%%



%%%%%%%%%%%%%%%%%%%%%%%%%%%%%%%%%%%%%%%%%%%%%%%%%%%%%%%%%%%%

\begin{frame}[fragile]{Matplotlib: boxplot}

El boxplot resume una distribución mediante cuartiles y valores extremos.

\begin{itemize}
    \item Se usa para comparar las distribuciones de diversas categorías en función de sus cuartiles y outliers.
\end{itemize}

Requiere un atributo categorico (eje x) y un atributo cuantitativo (eje y). 

\vspace{3mm}


 \begin{columns}

    \column{0.5\textwidth}


\begin{minted}{python}
>>> temperatura_santiago = np.random.randn(100) + 25
>>> plt.boxplot(temperatura_santiago)
>>> plt.ylabel("Valor [°C]")
>>> plt.xlabel("Temperatura en Santiago")
>>> plt.show()
\end{minted}

    \column{0.5\textwidth}

\begin{figure}[H]
\centering
\includegraphics[width=0.6\linewidth]{days/03/images/plt_boxplot.png}
\caption{Boxplot en matplotlib.}
\end{figure}

\end{columns}


\end{frame}

%%%%%%%%%%%%%%%%%%%%%%%%%%%%%%%%%%%%%%%%%%%%%%%%%%%%%%%%%%%%


\begin{frame}[fragile]{Matplotlib: imshow}

\mintinline{python}{imshow} se utiliza para visualizar datos matriciales como imágenes.

\begin{itemize}
    \item Es util cuando queremos visualizar imágenes, matrices, mapas, entre otros.
\end{itemize}

Recibe una matriz con valores numericos como argumento.

\vspace{3mm}



 \begin{columns}

    \column{0.5\textwidth}


\begin{minted}{python}
import numpy as np
# matrix de 10 filas x 10 columnas
matriz = np.random.rand(10, 10)
plt.imshow(matriz)
plt.colorbar()
plt.show()
\end{minted}

    \column{0.5\textwidth}

\begin{figure}[H]
\centering
\includegraphics[width=0.7\linewidth]{days/03/images/plt_imshow.png}
\caption{Imshow plot en matplotlib.}
\end{figure}

\end{columns}

\end{frame}



\begin{frame}[fragile]{Aplicación sobre Iris Dataset}

\begin{itemize}
   \item Dataset con 150 datos tabulares.
   \item Cada instancia (fila) representa una flor.
   \item En este dataset tiene 5 columnas: \textbf{sepallength, sepalwidth, petallength, petalwidth, class}.
   \item Las clases (class) son categoricas: \textbf{setosa, versicolor, virginica}.
\end{itemize}

\begin{figure}[H]
\centering
\includegraphics[width=0.5\linewidth]{days/03/images/iris.png}
\caption{Features en Iris Dataset.}
\end{figure}

\tiny \url{https://colab.research.google.com/drive/1kKzcxZf8Ova4Rd2Yrb-Zui5163hWYb6G?usp=sharing}

\end{frame}

%%%%%%%%%%%%%%%%%%%%%%%%%%%%%%%%%%%%%%%%%%%%%%%%%%%%%%%%%%%%
