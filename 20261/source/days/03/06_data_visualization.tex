\documentclass[aspectratio=169,handout]{beamer}
\usepackage{borelian}
\usepackage{multirow}

\begin{document}
    \classtitle{6}{Visualización de información}{7 de enero de 2026}

    \begin{frame}{Visualización de información}
        \begin{itemize}
            \item La información puede almacenarse en diferentes formatos: \textbf{grafos}, \textbf{árboles}, \textbf{tablas}, \textbf{diccionarios}, entre otros.
            \pause
            \item En este curso, nos enfocaremos en datos tabulares (tablas). Este es el formato más simple y utilizado.
            \pause
            \item En la clase anterior, vimos cómo abrir un conjunto de datos tabular usando \texttt{pandas}. Ahora, la idea es visualizar estos datos.
            \pause 
        \end{itemize}
        Pero antes de eso, ¿por qué nos gustaria visualizar los datos? 
        \begin{itemize}
            \item Como vimos en la clase de estadística, los conjuntos de datos pueden ser bastante complejos para analizar fila por fila. 
            \pause
            \item Las técnicas de visualización nos permiten encontrar patrones o tendencias en los datos.
            \pause
            \item La visualización de información nos permite responder a preguntas asociadas a los datos.
        \end{itemize}
    \end{frame}

    \begin{frame}[fragile]{\texttt{matplotlib} vs \texttt{seaborn}}
        En esta clase, trabajaremos con dos paquetes para visualizar datos.
        \begin{itemize}
            \item \texttt{matplotlib}: librería de uso general para construir todo tipo de gráficos.
            \begin{itemize}
                \item Es muy flexible, pero requiere más código.
                \item Recordar la importación: \mintinline{python}{import matplotlib.pyplot as plt}.
                \item Las funciones de \texttt{matplotlib} reciben los vectores de datos a graficar como argumentos.
            \end{itemize}

            \item \texttt{seaborn}: librería orientada al análisis de datos.
            \begin{itemize}
                \item Está pensada para trabajar directamente con \emph{DataFrames}.
                \item Recordar la importación: \mintinline{python}{import seaborn as sns}.
            \end{itemize}
        \end{itemize}
    \end{frame}

    \begin{frame}[fragile]{\texttt{matplotlib} vs \texttt{seaborn}}
        En \texttt{matplotlib}, los gráficos se construyen indicando:
        \begin{itemize}
            \item Directamente los \textbf{datos a graficar} (arreglos, listas, u objetos de tipo \mintinline{python}{pd.Series}).
            \item No existe un argumento \mintinline{python}{data} que agrupe los datos.
            \item Los argumentos (\mintinline{python}{x}, \mintinline{python}{y}, etc.) corresponden a los \textbf{arreglos de valores} y no a los nombres de las columnas.
        \end{itemize}
        En \texttt{seaborn}, los gráficos se construyen indicando:
        \begin{itemize}
            \item \mintinline{python}{data}: el \emph{DataFrame} que contiene los datos.
            \item El resto de los argumentos (\mintinline{python}{x}, \mintinline{python}{y}, \mintinline{python}{hue}, etc.) son los \textbf{nombres de las columnas} del \emph{DataFrame} que se quieren graficar.
        \end{itemize}
        Esto permite crear gráficos de forma simple, clara y eficiente, sin necesidad de extraer manualmente los datos como ocurre habitualmente en \texttt{matplotlib}.
    \end{frame}

    \begin{frame}[fragile]{Gráficos de línea}
        \begin{itemize}
            \item Los ejes $x$ e $y$ representan los valores que adoptan las variables analizadas.
            \item Se usan para analizar tendencias temporales y comparación de la evolución de variables.
            \item Las marcas son puntos que se conectan por líneas.
            \item La posición vertical expresa un valor cuantitativo, mientras la posición horizontal contiene las llaves ordenadas.
        \end{itemize}

        \begin{columns}[c]
            \column{0.5\linewidth}
            En la figura de la derecha, se indica la evolución de la temperatura en un año.

            \hfill
            \column{0.5\linewidth}
            \begin{figure}[H]
                \centering
                \includegraphics[width=\linewidth]{days/03/images/line_example.png}
            \end{figure}
        \end{columns}
    \end{frame}

    \begin{frame}[fragile]{Ejemplo de gráfico de línea (\texttt{matplotlib})}
        \begin{columns}
            \column{0.5\textwidth}
            Con \texttt{plt.plot}:
            \begin{minted}{python}
                >>> import matplotlib.pyplot as plt
                >>> x = [0, 1, 2, 3, 4]
                >>> y = [1, 3, 2, 5, 4]
                >>> # (0, 1), (1, 3), ... (4, 4)
                >>> plt.plot(x, y)
                >>> plt.ylabel("y")
                >>> plt.show()
            \end{minted}

            \hfill
            \column{0.5\textwidth}
            \begin{figure}[H]
                \centering
                \includegraphics[width=0.8\linewidth]{days/03/images/plt_line.png}
                \caption{Gráfico de línea en \texttt{matplotlib}. Muestra los valores de la lista $x$ (eje $x$) y lista $y$ (eje $y$).}
            \end{figure}
        \end{columns}
        Más ejemplos los pueden revisar en \href{https://matplotlib.org/stable/api/_as_gen/matplotlib.pyplot.plot.html}{este enlace}.
    \end{frame}


    \begin{frame}[fragile]{Ejemplo de gráfico de línea (\texttt{seaborn})}
        \begin{columns}
            \column{0.4\textwidth}
            Con \texttt{sns.lineplot}:
            \begin{minted}[fontsize=\small]{python}
                >>> import seaborn as sns
                >>> # DataFrame de vuelos
                >>> df = sns.load_dataset("flights")
                >>> sns.lineplot(data=df,
                ...              x="year",
                ...              y="passengers")
                >>> plt.show()
            \end{minted}

            \hfill
            \column{0.6\textwidth}
            \begin{figure}[H]
                \centering
                \includegraphics[width=0.75\linewidth]{days/03/images/lineplot_sea.png}
                \caption{Gráfico de línea que muestra el número de vuelos (eje $y$) según el año (eje $x$). El área sombreada representa un intervalo de confianza.}
            \end{figure}
        \end{columns}
        Más ejemplos los pueden revisar en \href{https://seaborn.pydata.org/generated/seaborn.lineplot.html}{este enlace}.
    \end{frame}

    \begin{frame}{Gráfico de dispersión}
        \begin{columns}
            \column{0.5\textwidth}
            \begin{itemize}
                \item El gráfico de dispersión muestra la relación entre dos variables numéricas.
                \item Permite detectar correlaciones, identificar patrones, clústeres y encontrar \emph{outliers}.
                \item Los canales son las posiciones: horizontal y vertical. Los ejes $x$ e $y$ representan los valores que adoptan las variables analizadas.
            \end{itemize}

            \hfill
            \column{0.5\textwidth}
            \begin{figure}[H]
                \centering
                \includegraphics[width=0.8\linewidth]{days/03/images/scatter_mpg.png}
                \caption{Gráfico de dispersión entre millas por galón (\texttt{MPG}, eje $y$) y caballos de fuerza (\texttt{horsepower}, eje $x$) en autos.}
            \end{figure}
        \end{columns}
    \end{frame}

    \begin{frame}[fragile]{Ejemplo de gráfico de dispersión (\texttt{matplotlib})}
        \begin{columns}
            \column{0.5\textwidth}
            Con \texttt{plt.scatter}:
            \begin{minted}[fontsize=\small]{python}
                >>> import matplotlib.pyplot as plt
                >>> x = [1, 2, 3, 4, 5]
                >>> y = [2, 1, 3, 5, 4]
                >>> # (1, 2), (2, 1), ... , (5, 4)
                >>> plt.scatter(x, y)
                >>> plt.ylabel("y")
                >>> plt.xlabel("x")
                >>> plt.show()
            \end{minted}
            
            \hfill
            \column{0.5\textwidth}
            \begin{figure}[H]
                \centering
                \includegraphics[width=0.8\linewidth]{days/03/images/plt_scatter.png}
                \caption{Gráfico de dispersión en \texttt{matplotlib}. Muestra los valores de la lista $x$ (eje $x$) y lista $y$ (eje $y$).}
            \end{figure}
        \end{columns}
        Más ejemplos los pueden revisar en \href{https://matplotlib.org/stable/api/_as_gen/matplotlib.pyplot.scatter.html}{este enlace}.
    \end{frame}

    \begin{frame}[fragile]{Ejemplo de gráfico de dispersión (\texttt{seaborn})}
        \begin{columns}
            \column{0.4\textwidth}
            Con \texttt{sns.scatterplot}:
            \begin{minted}[fontsize=\small]{python}
                >>> import seaborn as sns
                >>> # DataFrame de propinas
                >>> df = sns.load_dataset("tips")
                >>> sns.scatterplot(data=df,
                ...                 x="total_bill",
                ...                 y="tip")
                >>> plt.show()
            \end{minted}

            \hfill
            \column{0.6\textwidth}
            \begin{figure}[H]
                \centering
                \includegraphics[width=0.7\linewidth]{days/03/images/scatter_plot.png}
                \caption{Gráfico de dispersión entre pago total (\texttt{total bill}, eje $x$) y propina (\texttt{tip}, eje $y$).}
            \end{figure}
        \end{columns}

        Más ejemplos los pueden revisar en \href{https://seaborn.pydata.org/generated/seaborn.scatterplot.html}{este enlace}.
    \end{frame}

    \begin{frame}[fragile]{Gráfico de barras}
        \begin{itemize}
            \item Requiere 1 atributo categórico y 1 cuantitativo.
            \item Los canales que codifican información incluyen: el largo de la barra para expresar un valor cuantitativo y una separación en el espacio para representar otra categoría.
            \item Se puede usar para comparar categorías y encontrar casos extremos.
        \end{itemize}

        \begin{figure}[H]
            \centering
            \includegraphics[width=0.4\linewidth]{days/03/images/barplot.png}
            \caption{Edad promedio según la clase en el barco Titanic.}
        \end{figure}
    \end{frame}

    \begin{frame}[fragile]{Ejemplo de gráfico de barras (\texttt{matplotlib})}
        \begin{columns}
            \column{0.5\textwidth}
            Con \texttt{plt.bar}:
            \begin{minted}[fontsize=\small]{python}
                >>> import matplotlib.pyplot as plt
                >>> categorias = ["A", "B", "C", "D"]
                >>> valores = [23, 45, 12, 30]
                >>> plt.bar(categorias, valores)
                >>> plt.ylabel("Valor")
                >>> plt.xlabel("Categoría")
                >>> plt.show()
            \end{minted}

            \hfill
            \column{0.5\textwidth}
            \begin{figure}[H]
                \centering
                \includegraphics[width=0.7\linewidth]{days/03/images/plt_bar.png}
                \caption{Gráfico de barras generado en \texttt{matplotlib}. Muestra valores (eje $y$) para cada categoria (eje $x$).}
            \end{figure}
        \end{columns}
        Más ejemplos los pueden revisar en \href{https://matplotlib.org/stable/api/_as_gen/matplotlib.pyplot.bar.html}{este enlace}.
    \end{frame}


    \begin{frame}[fragile]{Ejemplo de gráfico de barras (\texttt{seaborn})}
        \begin{columns}
            \column{0.4\textwidth}
            Con \texttt{sns.barplot}:
            \begin{minted}[fontsize=\small]{python}
                >>> import seaborn as sns
                >>> # DataFrame de pingüinos
                >>> df = sns.load_dataset("penguins")
                >>> sns.barplot(df,
                                x="island",
                                y="body_mass_g")
                >>> plt.show()
            \end{minted}

            \hfill
            \column{0.6\textwidth}
            \begin{figure}[H]
                \centering
                \includegraphics[width=0.77\linewidth]{days/03/images/barplot_seaborn.png}
                \caption{Gráfico de barras mostrando el peso promedio de pingüinos (variable numérica en eje $y$) en islas (variable categórica).}
            \end{figure}
        \end{columns}
        Más ejemplos los pueden revisar en \href{https://seaborn.pydata.org/generated/seaborn.barplot.html}{este enlace}.
    \end{frame}

    \begin{frame}[fragile]{Histograma}
        \begin{columns}
            \column{0.5\textwidth}
            \begin{itemize}
                \item Un histograma representa una variable cuantitativa agrupando los datos en intervalos.
                \item Cada barra corresponde a un rango de valores del eje horizontal.
                \item Permite analizar la forma de la distribución, su asimetría y concentración.
                \item El eje $y$ muestra la frecuencia (número de observaciones) en cada intervalo.
            \end{itemize}

            \hfill
            \column{0.5\textwidth}
            \begin{figure}[H]
                \centering
                \includegraphics[width=0.6\linewidth]{days/03/images/hist.png}
                \caption{Histograma que muestra la distribución del largo del sépalo en el conjunto de datos Iris.}
            \end{figure}
        \end{columns}
    \end{frame}

    \begin{frame}[fragile]{Ejemplo de histograma (\texttt{matplotlib})}
        \begin{columns}
            \column{0.55\textwidth}
            Con \texttt{plt.hist}:
            \begin{minted}[fontsize=\small]{python}
                >>> import matplotlib.pyplot as plt
                >>> temperatura = [2, 33, 24, 25, 12, 20, 0, 23] 
                >>> plt.hist(temperatura)
                >>> plt.xlabel("Temperatura [°C]")
                >>> plt.ylabel("Conteo [-]")
                >>> plt.show()
            \end{minted}

            \hfill
            \column{0.45\textwidth}
            \begin{figure}[H]
                \centering
                \includegraphics[width=0.8\linewidth]{days/03/images/histograma_plt.png}
                \caption{Histograma en \texttt{matplotlib}. Muestra la frecuencia (eje $y$) de veces que ocurren valores de temperatura (eje $x$).}
            \end{figure}
        \end{columns}
        Más ejemplos los pueden revisar en \href{https://matplotlib.org/stable/api/_as_gen/matplotlib.pyplot.hist.html}{este enlace}.
    \end{frame}

    \begin{frame}[fragile]{Ejemplo de histograma (\texttt{seaborn})}
        \begin{columns}
            \column{0.5\textwidth}
            Con \texttt{sns.histplot}:
            \begin{minted}[fontsize=\small]{python}
                >>> import seaborn as sns
                >>> # DataFrame de pingüinos
                >>> df = sns.load_dataset("penguins") 
                >>> sns.histplot(data=df,
                ...              x="flipper_length_mm")
                >>> plt.show()
            \end{minted}

            \hfill
            \column{0.5\textwidth}
            \begin{figure}[H]
                \centering
                \includegraphics[width=0.75\linewidth]{days/03/images/hist_flipper.png}
                \caption{Histograma que muestra la frecuencia (eje $y$) de los distintos largos de aleta (\texttt{flipper\_length\_mm}) observados (eje $x$).}
            \end{figure}
        \end{columns}
        Más ejemplos los pueden revisar en \href{https://seaborn.pydata.org/generated/seaborn.histplot.html}{este enlace}.
    \end{frame}

    \begin{frame}[fragile]{\emph{Boxplot}}
        \begin{columns}
            \column{0.5\textwidth}
            \begin{itemize}
                \item El \emph{boxplot} resume una distribución considerando 5 valores obtenidos de la serie de datos: mínimo, cuartil 1 ($Q_1$), cuartil 2 ($Q_2$: mediana), cuartil 3 ($Q_3$) y máximo.
                \item Se usa para comparar las distribuciones entre columnas de un \emph{DataFrame} en función de sus cuartiles y \emph{outliers}.
                \item Requiere un atributo categórico (eje $x$) y un atributo cuantitativo (eje $y$). 
                \item El principal canal de codificación es el largo de la caja (rango intercuartil) y los puntos fuera de los límites (asociados a \emph{outliers}).
            \end{itemize}

            \hfill
            \column{0.5\textwidth}
            \begin{figure}[H]
                \centering
                \includegraphics[width=0.55\linewidth]{days/03/images/boxplot_draw.png}
                \caption{Componentes de \emph{boxplot}. IQR es el rango intercuartil.}
            \end{figure}
        \end{columns}
    \end{frame}

    \begin{frame}[fragile]{Ejemplo de \emph{boxplot} en \texttt{matplotlib}}
        \begin{columns}
            \column{0.5\textwidth}
            Con \texttt{plt.boxplot}:
            \begin{minted}[fontsize=\small]{python}
                >>> import matplotlib.pyplot as plt
                >>> temperatura = [2, 33, 24, 25, 12] + \
                >>> ...           [20, 0, 23, 50]
                >>> plt.boxplot(temperatura)
                >>> plt.ylabel("Temperatura")
                >>> plt.show()
            \end{minted}

            \hfill
            \column{0.5\textwidth}
            \begin{figure}[H]
                \centering
                \includegraphics[width=0.7\linewidth]{days/03/images/boxplot_plt.png}
                \caption{\emph{Boxplot} de la variable temperatura (eje $y$), que muestra el valor mínimo, el máximo y los cuartiles primero ($Q_1$), segundo ($Q_2$: mediana) y tercero ($Q_3$).}
            \end{figure}
        \end{columns}
        Más ejemplos los pueden revisar en \href{https://matplotlib.org/stable/api/_as_gen/matplotlib.pyplot.boxplot.html}{este enlace}.
    \end{frame}

    \begin{frame}[fragile]{Ejemplo de \emph{boxplot} en \texttt{seaborn}}
        \begin{columns}
            \column{0.5\textwidth}
            Con \texttt{sns.boxplot}:
            \begin{minted}[fontsize=\small]{python}
                >>> import seaborn as sns
                >>> df = sns.load_dataset('titanic')
                >>> sns.boxplot(data=df, x="class", y="age")
                >>> plt.ylabel("Edad")
                >>> plt.xlabel("Clase")
                >>> plt.show()
            \end{minted}

            \hfill
            \column{0.5\textwidth}
            \begin{figure}[H]
                \centering
                \includegraphics[width=0.6\linewidth]{days/03/images/box_seaborn.png}
                \caption{\emph{Boxplot} de distribuciones de edades según clase en embarcación Titanic. Muestra los valores de edades mínimo, máximo y los cuartiles primero ($Q_1$), segundo ($Q_2$: mediana) y tercero ($Q_3$) separados por clase.}
            \end{figure}
        \end{columns}
        Más ejemplos los pueden revisar en \href{https://seaborn.pydata.org/generated/seaborn.boxplot.html}{este enlace}.
    \end{frame}

    \begin{frame}[fragile]{\emph{Heatmaps}}
        \begin{columns}
            \column{0.5\textwidth}
            \begin{itemize}
                \item Un \emph{heatmap} es una visualización que representa valores numéricos mediante una escala de colores.
                \item Se utiliza para analizar patrones, concentraciones y relaciones en datos bidimensionales.
                \item Permite identificar rápidamente zonas de valores altos y bajos.
                \item Es comúnmente usado para visualizar matrices de correlación, distancias o frecuencias.
                \item El color es el principal canal de codificación del valor.
            \end{itemize}

            \hfill
            \column{0.5\textwidth}
            \begin{figure}[H]
                \centering
                \includegraphics[width=0.65\linewidth]{days/03/images/heatmap.png}
                \caption{Ejemplo de \emph{heatmap}. El eje $x$ representa a diferentes granjeros; el eje $y$ representa verduras. El color en la matriz representa la cantidad de verduras cosechadas (en toneladas por año) para cada combinación.}
            \end{figure}
        \end{columns}
    \end{frame}

    \begin{frame}[fragile]{\emph{Heatmap} con \texttt{matplotlib} (\texttt{imshow})}
        \begin{columns}
            \column{0.5\textwidth}
            Con \texttt{plt.imshow}:
            \begin{minted}[fontsize=\small]{python}
                >>> import matplotlib.pyplot as plt
                >>> import numpy as np
                >>> data = np.random.rand(4, 4)
                >>> plt.imshow(data, cmap="viridis")
                >>> plt.colorbar(label="Valor")
                >>> plt.xlabel("Indice columna")
                >>> plt.ylabel("Indice fila")
                >>> plt.title("Heatmap usando imshow")
                >>> plt.show()
            \end{minted}

            \hfill
            \column{0.5\textwidth}
            \vspace{-1cm}
            \begin{figure}[H]
                \centering
                \includegraphics[width=0.75\linewidth]{days/03/images/heatmap_plt.png}
                \caption{\emph{Heatmap} generado con \texttt{imshow} en \texttt{matplotlib}. Los colores representan los valores numéricos de una matriz de $4 \times 4$ generada aleatoriamente, donde cada celda se codifica mediante una escala de color.}
            \end{figure}
        \end{columns}
        Más ejemplos los pueden revisar en \href{https://matplotlib.org/stable/api/_as_gen/matplotlib.pyplot.imshow.html}{este enlace}.
    \end{frame}

    \begin{frame}[fragile]{\emph{Heatmap} con \texttt{seaborn}}
        \begin{columns}
            \column{0.45\textwidth}
            Con \texttt{sns.heatmap}:
            \begin{minted}[fontsize=\small]{python}
                >>> import seaborn as sns
                >>> df = sns.load_dataset("iris")
                >>> corr = df.corr(numeric_only=True)
                >>> sns.heatmap(corr, annot=True,
                ...             cmap="coolwarm")
                >>> plt.title("Matriz de correlación")
                >>> plt.show()
            \end{minted}
            Más ejemplos los pueden revisar en \href{https://seaborn.pydata.org/generated/seaborn.heatmap.html}{este enlace}.

            \hfill
            \column{0.55\textwidth}
            \vspace{-1cm}
            \begin{figure}[H]
                \centering
                \includegraphics[width=0.8\linewidth]{days/03/images/correlation_seaborn.png}
                \caption{\emph{Heatmap} de la matriz de correlación del dataset Iris, donde los colores indican la intensidad y el signo de la correlación lineal entre las variables numéricas, y los valores anotados corresponden a los coeficientes de correlación.}
            \end{figure}
        \end{columns}
    \end{frame}

    \begin{frame}{Ejemplo práctico}
        Exploraremos un conjunto de datos que tiene información sobre Pokémon en \href{https://colab.research.google.com/drive/1Tdz4RbEasHZ69BbEkGKErjuKW-TIuN90?usp=sharing}{Google Colab}, para aplicar lo aprendido en esta clase.
        \begin{figure}[H]
            \centering
            \includegraphics[width=0.5\linewidth]{days/03/images/pokemon.png}
        \end{figure}
    \end{frame}
    
    \begin{frame}[fragile]{\texttt{matplotlib} (Resumen)}
        \begin{minted}{python}
            import matplotlib.pyplot as plt # Importación
        \end{minted}
        \begin{table}[H]
            \centering
            \begin{tabular}{c|c|c}
                Función & Uso & Ejemplo \\
                \hline
                \mintinline{python}{plot} & Gráfico de línea & \mintinline{python}{plt.plot(valores)} \\
                \hline
                \mintinline{python}{scatter} & Gráfico de puntos & \mintinline{python}{plt.scatter(valores_x, valores_y)} \\
                \hline
                \mintinline{python}{bar} & Gráfico de barras & \mintinline{python}{plt.bar(etiquetas, valores)} \\
                \hline
                \mintinline{python}{boxplot} & Gráfico de caja & \mintinline{python}{plt.boxplot(valores)} \\
                \hline
                \mintinline{python}{histogram} & Histograma & \mintinline{python}{plt.hist(valores)} \\
                \hline
                \mintinline{python}{imshow} & Gráfico \emph{heatmap} & \mintinline{python}{plt.imshow(matriz)}
            \end{tabular}
            \caption{Funciones en \texttt{matplotlib}.}
        \end{table}
    \end{frame}

    \begin{frame}[fragile]{\texttt{seaborn} (Resumen)}
        \begin{minted}{python}
            import seaborn as sns # Importación
        \end{minted}
        \begin{itemize}
            \item \texttt{lineplot}: gráfico de líneas entre dos columnas de un \emph{DataFrame}. \\  
            \mintinline{python}{sns.lineplot(data=dataframe, x="x", y="y")}

            \item \texttt{scatterplot}: gráfico de dispersión entre dos columnas de un \emph{DataFrame}. \\
            \mintinline{python}{sns.scatterplot(data=dataframe, x="x", y="y")}

            \item \texttt{histplot}: histograma de una columna de un \emph{DataFrame}. \\
            \mintinline{python}{sns.histplot(data=dataframe, x="x")}

            \item \texttt{boxplot}: comparación entre distribuciones de variables mirando sus cuartiles. \\ 
            \mintinline{python}{sns.boxplot(data=dataframe, x="grupo", y="valor")}

            \item \texttt{heatmap}: visualización matricial donde los colores representan los valores numéricos de una tabla, comúnmente usada para analizar relaciones entre variables (p. ej., matrices de correlación). \\
            \mintinline{python}{sns.heatmap(dataframe, annot=True)}
        \end{itemize}
    \end{frame}

    \begin{frame}{Referencias}
        \nocite{mckinney2022}
        \bibliographystyle{plainnat}
        \bibliography{../../references}
    \end{frame}
\end{document}