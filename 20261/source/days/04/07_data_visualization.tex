%Visualización de información. Gráficos (barras, histogramas, líneas, dispersión, etc.) y sus objetivos. Uso de colores e ilusiones ópticas.

\documentclass[11pt,t,aspectratio=169]{beamer}
\usepackage{borelian}
\usepackage{minted}

\begin{document}
\classtitle{7}{Visualización de Datos en Python.}

%%%%%%%
\begin{frame}{Contenidos de hoy}
    
    \begin{itemize}%[<+->]
       
        \item Matplotlib Plots
        
        \item Seaborn Plots
        

    \end{itemize}
    
\end{frame}




\begin{frame}{Visualización de Información}

La visualización de datos es una parte esencial en estudios sobre Inteligencia Artificial. En esta clase explicaremos funciones utiles de dos librerías.

%https://matplotlib.org/stable/users/explain/quick_start.html#quick-start


\vspace{5mm}

\begin{itemize}
    \item $\code{Matplotlib}$
    \item $\code{Seaplot}$
\end{itemize}
\end{frame}


\begin{frame}[fragile]{Matplotlib}

Matplotlib es la librería mas popular para visualizar datos en Python. Esta libreria tiene funciones acopladas al paquete de Pandas.

\begin{minted}{python}
import matplotlib.pyplot as plt
\end{minted}

\begin{table}[]
    \centering
    \begin{tabular}{c|c|c}
    Función & Uso & Ejemplo  \\
    \hline
    $\code{plot}$ & Gráfico de linea & $\code{ plt.plot(valores) }$ \\
    \hline
    $\code{scatter}$ & Gráfico de puntos & $\code{ plt.scatter(valores) }$ \\
    \hline
    $\code{bar}$ & Gráfico de barras & $\code{ plt.bar(etiquetas, conteos) }$ \\
    \hline
    $\code{boxplot}$ & Gráfico de caja & $\code{ plt.boxplot(valores) }$ \\
    \hline
    $\code{histogram}$ & Histograma & $\code{ plt.hist(valores) }$ \\
    \hline
    $\code{imshow}$ & Histograma & $\code{ plt.plot(valores) }$ \\
    
    \end{tabular}
    \caption{Funciones en matplotlib.}
    \label{tab:placeholder}
\end{table}


\tiny \url{https://colab.research.google.com/drive/1X7xoTisfFrGZFfrQ9ROckTaYzv8Xw8TB?usp=sharing}

\end{frame}


\begin{frame}[fragile]{Seaborn}

Seaborn es una librería de visualización que permite generar gráficos de manera simple, rápida y con muchos estilos.
\begin{minted}{python}
import seaborn as sns
\end{minted}

\begin{itemize}
    \item \textbf{lineplot}:  
    Gráfico de líneas con estimación estadística e intervalos de confianza. \\  
    \code{sns.lineplot(data=df, x="x", y="y")}

    \item \textbf{scatterplot}:  
    Gráfico de dispersión con agrupación por color, tamaño o estilo.  \\
    \code{sns.scatterplot(data=df, x="x", y="y", hue="clase")}

    \item \textbf{histplot}:  
    Histograma y densidad de una variable. \\
    \code{sns.histplot(data=df, x="x", kde=True)}

    \item \textbf{boxplot}:  
    Comparación de distribuciones mediante cajas. \\ 
    \code{sns.boxplot(data=df, x="grupo", y="valor")}

    \item \textbf{violinplot}:  
    Distribución completa combinando KDE y boxplot. \\
    \code{sns.violinplot(data=df, x="grupo", y="valor")}
\end{itemize}


\end{frame}

\begin{frame}[fragile]{Seaborn}

\begin{minted}{python}
import seaborn as sns
\end{minted}

\begin{itemize}
    \item \textbf{barplot}:  
    Promedios con intervalos de confianza. \\ 
    \code{sns.barplot(data=df, x="grupo", y="valor")}

    \item \textbf{countplot}:  
    Conteo de observaciones por categoría.\\  
    \code{sns.countplot(data=df, x="categoria")}

    \item \textbf{heatmap}:  
    Visualización matricial (correlaciones, tablas). \\ 
    \code{sns.heatmap(df.corr(), annot=True)}

    \item \textbf{pairplot}:  
    Relación entre todas las variables numéricas.\\  
    \code{sns.pairplot(df, hue="clase")}

    \item \textbf{jointplot}:  
    Relación bivariada con distribuciones marginales.\\  
    \code{sns.jointplot(data=df, x="x", y="y")}
\end{itemize}

\tiny Ejemplos!: \url{https://colab.research.google.com/drive/1TSchBjAdHT3Mx1MrvHSKp32o9HvGjk9n?usp=sharing}

\end{frame}





\begin{frame}{Referencias:}
    \begin{itemize}
        \item Wes McKinney. (2022). Python for Data Analysis. Third Edition.
        \item \url{https://seaborn.pydata.org/tutorial/function_overview.html}
        \item 
    \end{itemize}
\end{frame}


\end{document}