%Visualización de información. Gráficos (barras, histogramas, líneas, dispersión, etc.) y sus objetivos. Uso de colores e ilusiones ópticas.

\documentclass[11pt,t,aspectratio=169]{beamer}
\usepackage{borelian}

\usetheme{lehighlight}
\usefonttheme{professionalfonts}

\setbeamersize{
    text margin left=2em,
    text margin right=2em
}

\usepackage{hyperref}
\usepackage{multicol}
\usepackage{minted}
\usepackage{xcolor}  
\usepackage{colortbl}
\usepackage{graphics}
\usepackage{tikz}
\usepackage{ragged2e}
\usepackage{listings}

\usepackage{cmbright}
\usepackage[T1]{fontenc}


\usepackage{listings}

\usepackage{xcolor}

%New colors defined below
\definecolor{codegreen}{rgb}{0,0.6,0}
\definecolor{codegray}{rgb}{0.5,0.5,0.5}
\definecolor{codepurple}{rgb}{0.58,0,0.82}
\definecolor{backcolour}{rgb}{0.95,0.95,0.92}

%Code listing style named "mystyle"
\lstdefinestyle{mystyle}{
  backgroundcolor=\color{backcolour}, commentstyle=\color{codegreen},
  keywordstyle=\color{magenta},
  numberstyle=\tiny\color{codegray},
  stringstyle=\color{codepurple},
  basicstyle=\ttfamily\footnotesize,
  breakatwhitespace=false,         
  breaklines=true,                 
  captionpos=b,                    
  keepspaces=true,                 
  numbers=left,                    
  numbersep=5pt,                  
  showspaces=false,                
  showstringspaces=false,
  showtabs=false,                  
  tabsize=2
}

%"mystyle" code listing set
\lstset{style=mystyle}



\apptocmd{\frame}{}{\justifying}{}

\newcommand{\mpar}{\vspace{3mm}\par}
\newcommand{\complejidad}[1]{\mpar\textsf{\color{blue} Complejidad: $\mathcal{O}(#1)$}.}

\newcommand{\anotar}[1]{\vspace{1mm}{\footnotesize \color{blue} #1}}

\newcommand{\code}[1]{{\color{purple} \textbf{#1}}}

% Delete this, if you do not want the table of contents to pop up at the beginning of each subsection:
\AtBeginSection[]
{
    \begingroup
        \setbeamertemplate{background canvas}[vertical shading]
        \setbeamertemplate{footline}[sectionfootline] 
        \setbeamertemplate{section page}[mysection]
        \frame[c]{
        \sectionpage
        }
    \endgroup
}

\title{\Large Clase 3: Programación Dinámica}
\subtitle{\Large CC4002-CC4006}
\author{}
\institute{Universidad de Chile}
\date{\today}



\begin{document}
\classtitle{7}{Visualización de Datos en Python.}

%%%%%%%
\begin{frame}{Contenidos de hoy}
    
    \begin{itemize}%[<+->]
       
        \item Matplotlib Plots
        
        \item Seaborn Plots
        

    \end{itemize}
    
\end{frame}




\begin{frame}{Visualización de Información}

La visualización de datos es una parte esencial en estudios sobre Inteligencia Artificial. En esta clase explicaremos funciones utiles de dos librerías.

%https://matplotlib.org/stable/users/explain/quick_start.html#quick-start


\vspace{5mm}

\begin{itemize}
    \item $\code{Matplotlib}$
    \item $\code{Seaplot}$
\end{itemize}
\end{frame}


\begin{frame}{Matplotlib}

Matplotlib es la librería mas popular para visualizar datos en Python. Esta libreria tiene funciones acopladas al paquete de Pandas :).

\begin{table}[]
    \centering
    \begin{tabular}{c|c|c}
    Función & Uso & Ejemplo  \\
    \hline
    $\code{plot}$ & Gráfico de linea & $\code{ plt.plot(valores) }$ \\
    \hline
    $\code{scatter}$ & Grafico de puntos & $\code{ plt.scatter(valores) }$ \\
    \hline
    $\code{boxplot}$ & Grafico de caja & $\code{ plt.boxplot(valores) }$ \\
    \hline
    $\code{histogram}$ & Histograma & $\code{ plt.hist(valores) }$ \\
    \hline
    $\code{imshow}$ & Histograma & $\code{ plt.plot(valores) }$ \\
    
    \end{tabular}
    \caption{Funciones en matplotlib.}
    \label{tab:placeholder}
\end{table}


Veamos algunos ejemplos: \url{https://colab.research.google.com/drive/1X7xoTisfFrGZFfrQ9ROckTaYzv8Xw8TB?usp=sharing}

\end{frame}


\begin{frame}{Seaborn}

Seaborn es una de las librerias mas populares para visualizar datos.


\begin{table}[]
    \centering
    \begin{tabular}{c|c|c}
    Función & Uso & Ejemplo  \\
    \hline
    $\code{lineplot}$ & Gráfico de linea & $\code{ plt.plot(valores) }$ \\
    \hline
    $\code{scatter}$ & Grafico de puntos & $\code{ plt.scatter(valores) }$ \\
    \hline
    $\code{boxplot}$ & Grafico de caja & $\code{ plt.boxplot(valores) }$ \\
    \hline
    $\code{histogram}$ & Histograma & $\code{ plt.hist(valores) }$ \\
    \hline
    $\code{imshow}$ & Histograma & $\code{ plt.plot(valores) }$ \\
    
    \end{tabular}
    \caption{Funciones en matplotlib.}
    \label{tab:placeholder}
\end{table}


Veamos algunos ejemplos: \url{https://colab.research.google.com/drive/1X7xoTisfFrGZFfrQ9ROckTaYzv8Xw8TB?usp=sharing}

\end{frame}







\begin{frame}{Referencias:}
    \begin{itemize}
        \item Wes McKinney. (2022). Python for Data Analysis. Third Edition.
    \end{itemize}
\end{frame}


\end{document}