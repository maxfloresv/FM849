%Visualización de información. Gráficos (barras, histogramas, líneas, dispersión, etc.) y sus objetivos. Uso de colores e ilusiones ópticas.

\documentclass[aspectratio=169]{beamer}
\usepackage{borelian}

\begin{document}
    \classtitle{7}{Ingeniería de características}{8 de enero de 2026}

    \begin{frame}{Motivación}
        En general, los datos reales vienen con problemas como:
        \begin{itemize}
            \item Datos faltantes.
            \item Datos desordenados.
        \end{itemize}
        Es importante entregar datos no nulos y no repetidos a modelos, ya que estas situaciones pueden crear un sesgo. En esta clase, seguiremos aprendiendo algunas funciones útiles en \texttt{pandas} que nos permitirán modificar \emph{DataFrames} y realizar un análisis exploratorio sobre datos tabulares.
    \end{frame}

    \begin{frame}{Continuación del ejemplo práctico}
        Vamos a seguir explorando el conjunto de datos que tiene información sobre Pokémon en \href{https://colab.research.google.com/drive/1qzo9PRNI5_31AzmoNyVx1I22hJy5MCDf?usp=sharing}{Google Colab}.
        \begin{figure}[H]
            \centering
            \includegraphics[width=0.5\linewidth]{days/03/images/pokemon.png}
        \end{figure}
    \end{frame}

    \begin{frame}{Limpieza y preparación de datos}
        \begin{itemize}
            \item Tratamiento de valores faltantes.
            \begin{itemize}
                \item Detectar valores NA $\rightarrow$ \texttt{isna()}, \texttt{notna()}.
                \item Eliminar valores NA $\rightarrow$ \texttt{dropna()}, \texttt{fillna()}.
            \end{itemize}
            \texttt{NA} se refiere a valores nulos/no disponibles.
            
            \item Corrección de tipos de datos.
            \begin{itemize}
                \item Cambiar el tipo $\rightarrow$ \texttt{astype()}.
                \item Crear fechas $\rightarrow$ \texttt{to\_datetime()}.
            \end{itemize}

            \item Limpieza básica.
            \begin{itemize}
                \item Reemplazar valores $\rightarrow$ \texttt{replace()}.
                \item Identificar y remover duplicados $\rightarrow$ \texttt{duplicated()}, \texttt{drop\_duplicates()}.
            \end{itemize}
        \end{itemize}            
    \end{frame}

    \begin{frame}{Agregación de datos}
        \begin{itemize}
            \item Cálculo de estadísticas que resumen los datos.
            \begin{itemize}
                \item \texttt{mean()}, \texttt{sum()}, \texttt{count()}, \texttt{min()}, \texttt{max()}, \texttt{std()}.
            \end{itemize}

            \item Agregación por grupos.
            \begin{itemize}
                \item \texttt{agg()}.
            \end{itemize}

            \item Transformaciones agregadas.
            \begin{itemize}
                \item Transformar datos $\rightarrow$ \texttt{transform()}.
            \end{itemize}
        \end{itemize}
    \end{frame}

    \begin{frame}{Agrupamiento de datos}
        \begin{itemize}
            \item Agrupamiento por variables categóricas.
            \begin{itemize}
                \item \texttt{groupby()}.
            \end{itemize}

            \item Agrupamiento por intervalos.
            \begin{itemize}
                \item \texttt{cut()}.
                \item \texttt{qcut()}.
            \end{itemize}

            \item Reestructuración de datos.
            \begin{itemize}
                \item \texttt{pivot()}, \texttt{pivot\_table()}.
                \item \texttt{melt()}.
            \end{itemize}
        \end{itemize}
    \end{frame}

    \begin{frame}{Referencias}
        \nocite{mckinney2022}
        \bibliographystyle{plainnat}
        \bibliography{../../references}
    \end{frame}
\end{document}