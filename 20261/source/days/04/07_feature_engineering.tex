%Visualización de información. Gráficos (barras, histogramas, líneas, dispersión, etc.) y sus objetivos. Uso de colores e ilusiones ópticas.

\documentclass[11pt,t,aspectratio=169]{beamer}
\usepackage{borelian}

\begin{document}
\classtitle{7}{Ingeniería de Características.}

%%%%%%%
\begin{frame}{Contenidos de hoy}
    
    \begin{itemize}%[<+->]

       \item Limpieza y Preparacion de Datos.
       
       \item Agregacion de Datos.
       
       \item Agrupamiento de Datos. 
        

    \end{itemize}
    
\end{frame}

\begin{frame}{Motivación}
    En general, los datos reales vienen con problemas:
    \begin{itemize}
        \item Datos faltantes.
        \item Datos desordenados.
    \end{itemize}

    Es importante entregar datos no nulos y no repetidos a modelos, ya que estas situaciones pueden crear un sesgo. En esta clase seguiremos aprendiendo algunas funciones utiles en pandas que nos permitirán modificar dataframes y realizar análisis exploratorio sobre datos tabulares.
    
\end{frame}

\begin{frame}{Vamos a continuar con el ejemplo de la clase anterior}
    Vamos a continuar explorando el dataset de Pokemons: \url{https://colab.research.google.com/drive/1qzo9PRNI5_31AzmoNyVx1I22hJy5MCDf?usp=sharing}.

\begin{figure}
    \centering
    \includegraphics[width=0.5\linewidth]{days/03/image/pokemon.png}
    \caption{Pokemones de primera generación.}
    \label{fig:placeholder}
\end{figure}
    
\end{frame}

%%%%%%%%%%%%%%%%%%%%%%%%%%%%% RESUMEN %%%%%%%%%%%%%%%%%%%%%%%%%%%%%%%%%%%%%%%%%%%%%%

\begin{frame}{Limpieza y Preparación de Datos.}

\begin{itemize}
    \item \textbf{Tratamiento de valores faltantes}
    \begin{itemize}
        \item Detectar valores NA $\rightarrow$ \texttt{isna()}, \texttt{notna()}
        \item Eliminar valores NA $\rightarrow$ \texttt{dropna()}, \texttt{fillna()}
    \end{itemize}

    NA se refiere a valores nulos/no disponibles.
    
    \vspace{3mm}
    
    \item \textbf{Corrección de tipos de datos}
    \begin{itemize}
        \item Cambiar el tipo $\rightarrow$ \texttt{astype()}
        \item Crear fechas $\rightarrow$ \texttt{to\_datetime()}
    \end{itemize}
    
    \vspace{3mm}

    \item \textbf{Limpieza básica}
    
    \begin{itemize}
        \item Reemplazar valores: $\rightarrow$ \texttt{replace()}
        \item Remover duplicados: $\rightarrow$ \texttt{duplicated()}, \texttt{drop\_duplicates()}
    \end{itemize}
\end{itemize}
    
\end{frame}


%%%%%%%
\begin{frame}{Agregación de Datos}

\begin{itemize}
    \item \textbf{Cálculo de estadísticas resumen}
    \begin{itemize}
        \item Calcular estadisticas $\rightarrow$  \texttt{mean()}, \texttt{sum()}, \texttt{count()}
        \item Calcular estadisticas $\rightarrow$ \texttt{min()}, \texttt{max()}, \texttt{std()}
    \end{itemize}

    \vspace{3mm}

    \item \textbf{Agregación por grupos}
    \begin{itemize}
        \item Agregar datos $\rightarrow$ \texttt{agg()}
    \end{itemize}

    \vspace{3mm}

    \item \textbf{Transformaciones agregadas}
    \begin{itemize}
        \item Transformar datos $\rightarrow$ \texttt{transform()}
    \end{itemize}
\end{itemize}

\end{frame}


\begin{frame}{Agrupamiento de Datos}

\begin{itemize}
    \item \textbf{Agrupamiento por variables categóricas}
    \begin{itemize}
        \item \texttt{groupby()}
    \end{itemize}

    \item \textbf{Agrupamiento por intervalos}
    \begin{itemize}
        \item \texttt{cut()}
        \item \texttt{qcut()}
    \end{itemize}

    \item \textbf{Reestructuración de datos}
    \begin{itemize}
        \item \texttt{pivot()}, \texttt{pivot\_table()}
        \item \texttt{melt()}
    \end{itemize}
\end{itemize}

\end{frame}











\begin{frame}{Referencias:}
    \begin{itemize}
        \item Wes McKinney. (2022). Python for Data Analysis. Third Edition.
    \end{itemize}
\end{frame}


\end{document}