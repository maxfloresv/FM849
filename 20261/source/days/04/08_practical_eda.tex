%Sesión práctica. Análisis exploratorio de datos.
% https://www.kaggle.com/code/gulsahdemiryurek/video-game-sales-exploratory-data-analysis
%https://www.kaggle.com/code/georgyzubkov/covid-19-exploratory-data-analysis


\documentclass[11pt,t,aspectratio=169]{beamer}
\usepackage{borelian}

\begin{document}
\classtitle{7}{Análisis Exploratorio de Datos}

\begin{frame}{Análisis exploratorio de datos}
El análisis exploratorio de datos (EDA) es un procedimiento que permite visualizar y entender los datasets con los que trabajamos.    

\begin{figure}
    \centering
    \includegraphics[width=0.5\linewidth]{days/04/image/eda.png}
    \caption{Esquema resumen sobre Analisis Exploratorio de Datos.}
    \label{fig:placeholder}
\end{figure}

\end{frame}

\begin{frame}{Análisis exploratorio de datos}
    
    \begin{itemize}
        \item En esta clase aplicaremos todos los conceptos vistos hasta ahora en el curso. 
        
        \item Esto nos permitirá realizar un Análisis Exploratorio de datos sobre dataset de incendios en Portugal.
    \end{itemize}


    \url{https://colab.research.google.com/drive/1cjp2M3dDSKtO1k-D9iZ1Y8VikvBKSTH5?usp=sharing}
\end{frame}


\end{document}