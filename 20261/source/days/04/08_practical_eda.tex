\documentclass[aspectratio=169]{beamer}
\usepackage{borelian}

\begin{document}
    \classtitle{8}{Análisis exploratorio de datos (sesión práctica)}{8 de enero de 2026}

    \begin{frame}{Análisis exploratorio de datos}
        El análisis exploratorio de datos (en inglés, EDA, por sus siglas) es un procedimiento que permite visualizar y entender los conjuntos de datos con los que trabajamos.    
        \begin{figure}[H]
            \centering
            \includegraphics[width=0.5\linewidth]{days/04/images/eda.png}
            \caption{Esquema resumen sobre análisis exploratorio de datos.}
        \end{figure}
    \end{frame}

    \begin{frame}{Análisis exploratorio de datos}
        \begin{itemize}
            \item En esta clase, aplicaremos todos los conceptos vistos hasta ahora en el curso. 
            \item Esto nos permitirá realizar un análisis exploratorio de datos sobre un conjunto de datos de bicicletas rentadas en Seúl.
        \end{itemize}
        Pueden acceder al \emph{notebook} de la sesión práctica en el \href{https://colab.research.google.com/drive/1cjp2M3dDSKtO1k-D9iZ1Y8VikvBKSTH5?usp=sharing}{siguiente enlace}.
    \end{frame}
\end{document}