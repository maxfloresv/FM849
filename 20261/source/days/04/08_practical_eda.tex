%Sesión práctica. Análisis exploratorio de datos.
% https://www.kaggle.com/code/gulsahdemiryurek/video-game-sales-exploratory-data-analysis
%https://www.kaggle.com/code/georgyzubkov/covid-19-exploratory-data-analysis


\documentclass[11pt,t,aspectratio=169]{beamer}
\usepackage{borelian}

\begin{document}
\classtitle{8}{Análisis Exploratorio de Datos - Sesión Práctica}

\begin{frame}{Análisis exploratorio de datos}
El análisis exploratorio de datos (EDA) es un procedimiento que permite visualizar y entender los datasets con los que trabajamos.    

\begin{figure}
    \centering
    \includegraphics[width=0.5\linewidth]{days/04/image/eda.png}
    \caption{Esquema resumen sobre Análisis Exploratorio de Datos.}
    \label{fig:placeholder}
\end{figure}

\end{frame}

\begin{frame}{Análisis exploratorio de datos}
    
    \begin{itemize}
        \item En esta clase aplicaremos todos los conceptos vistos hasta ahora en el curso. 
        
        \item Esto nos permitirá realizar un Análisis Exploratorio de datos sobre un dataset de bicicletas rentadas en Seoul.
    \end{itemize}


    \url{https://colab.research.google.com/drive/1cjp2M3dDSKtO1k-D9iZ1Y8VikvBKSTH5?usp=sharing}
\end{frame}


\end{document}



%\item \textbf{violinplot}:  
%Distribución completa combinando KDE y boxplot. \\
%\code{sns.violinplot(data=df, x="grupo", y="valor")}

%\begin{frame}[fragile]{Seaborn}
%\begin{minted}{python}
%import seaborn as sns
%\end{minted}
%\begin{itemize}
%    \item \textbf{barplot}:  
%    Promedios con intervalos de confianza. \\ 
%    \code{sns.barplot(data=df, x="grupo", y="valor")}

%    \item \textbf{countplot}:  
%    Conteo de observaciones por categoría.\\  
%    \code{sns.countplot(data=df, x="categoria")}

%    \item \textbf{heatmap}:  
%    Visualización matricial (correlaciones, tablas). \\ 
%    \code{sns.heatmap(df.corr(), annot=True)}

    %\item \textbf{jointplot}:  
    %Relación bivariada con distribuciones marginales.\\  
    %\code{sns.jointplot(data=df, x="x", y="y")}
%\end{itemize}
%\tiny Mas ejemplos!: \url{https://colab.research.google.com/drive/1TSchBjAdHT3Mx1MrvHSKp32o9HvGjk9n?usp=sharing}
%\end{frame}
